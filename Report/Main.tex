% !TEX encoding = UTF-8 Unicode
\documentclass[a4paper,twocolumn,twoside]{article}
%%%%%%%%%%%%%%%%%%%%%%%%%%%%%%%%%%%%%%%%%%%%%%%%%%%%%%%%%%%%%%%%%%%
% Packages using
%%%%%%%%%%%%%%%%%%%%%%%%%%%%%%%%%%%%%%%%%%%%%%%%%%%%%%%%%%%%%%%%%%%
\ifx\pdfoutput\undefined
	\usepackage[dvips]{graphicx}
	\DeclareGraphicsExtensions{.eps}
\else
	\usepackage[pdftex]{graphicx}
	\DeclareGraphicsExtensions{.pdf,.jpg,.png,.mps}
	\pdfcompresslevel=9
\fi
\graphicspath{{./Figures_for_report/}}
\usepackage[utf8]{inputenc}
\usepackage[english]{babel}
\usepackage{indentfirst}
\addtolength{\topmargin}{-20mm}
\addtolength{\textheight}{10mm}
\usepackage{amsmath} % Added to get modern math environments
\usepackage{amssymb,amsfonts} %Added to get math
\usepackage{amsthm} % Added to get theorems
\usepackage{natbib} % Added to get better bibliography
\usepackage{soul} %underline
\usepackage{url}
%\usepackage{listings}
%\usepackage{color}
% Special hack below to break the URL!
\def\UrlBreaks{\do\.\do\@\do\\\do\/\do\!\do\_\do\|\do\;\do\>\do\]%
 \do\)\do\,\do\?\do\'\do+\do\=\do\#\do\i\do\m\do\t\do\a\do\x}%
\urlstyle{rm} %
\usepackage[bookmarks=true,bookmarksnumbered=true,hypertexnames=true,breaklinks=true,colorlinks=true]{hyperref}
\hypersetup{
pdfauthor = {Torbj\"{o}rn E. M. Nordling},
pdftitle = {Information Retrieval and Processing--Setup of a Full Text System Implementing Automatic Metadata Extraction and Visualization},
pdfsubject = {Technical Report},
pdfkeywords = {Information retrieval, Metadata}}

% Headers and footers (must be after the document settings)
\usepackage{fancyhdr} %Custom header package
\pagestyle{fancy} %Turn on fancy headers
\fancyhead{} %Clears default layout
\fancyfoot{} %Clears default layout
\fancyhead[LO,RE]{\small \normalfont \leftmark} %Adds section headline to header
\fancyhead[LE,RO]{\slshape \rightmark} %Adds subsection headline to header
\fancyfoot[LO,RE]{\href{http://www.nordlinglab.org/ScientificInformation}{nordlinglab.org/ScientificInformation}}
\fancyfoot[RO,LE]{Information Retrieval and Processing}
\fancyfoot[CO,CE]{\thepage}
\renewcommand{\headrulewidth}{0.4pt} %Header line
\renewcommand{\footrulewidth}{0.4pt} %Footer line

% Select what to do with todonotes: 
% \usepackage[disable]{todonotes} % notes not showed
\usepackage[draft]{todonotes}   % notes showed

%%%%%%%%%%%%%%%%%%%%%%%%%%%%%%%%%%%%%%%%%%%%%%%%%%%%%%%%%%%%%%%%%%%

\begin{document} 
	
	\title{Information Retrieval and Processing--Setup of a Full Text System Implementing Automatic Metadata Extraction and Visualization}
	\author{Bernie Huang, Chinweze Ubadigha, Dexter Chen, Eric Chang, Eric Lee, \\
		Feng-Chun Hsia, Henry Peng, Hoang Tan, I-Chieh Lin, Jacky Wu, Jim Lan, \\
		Jones Hou, Karthick Mani, Kenny Hsu, Piyarul Hoque, Rahul Aditya, Ray \\
		Chang, Tan Phat, Wei, Yu-cheng Chen, Kenvin Lo, Dickson Lee, Rain Wu\\
		 Sareddy Reddy, Lewis Hsu, Paul Lin, Tam-Van Ngo, Torbj\"{o}rn E. M. Nordling}  % Please add your name here in alphabetic order (except Prof. N who is last)
	\maketitle   
	
	\section{Introduction}
	\label{Introduction}
	
	This technical report contains both general information on the state of the scientific information retrieval and processing art and a description of the \href{https://bitbucket.org/nordron/nordron-sciinfo}{Nordron-SciInfo} software package for information retrieval and processing. 
	
	This report was written by all participants of the \emph{Scientific Information Gathering and Processing for Engineering Research} course led by Prof. Nordling at the National Cheng Kung University starting from the spring semester of 2016.
	
	The main objective of the course is to teach the students state of the art information retrieval, processing methods, project management, and technical writing by doing project on implementation of some methods for retrieval and processing of full text scientific articles.
	
	The results of the student project is described in this report.
	
	\subsection{Aim and project structure}
	\label{aim}
	
	The final goal of this course is to construct an user friendly scientific article searching system with a critical sentence highlight function.
	
	To consider the demands of user, and describe the project goal in more details, the following user story was defined:
	
	As the researcher, We need a PDF reader where we can click on a highlighted text and thereby perform a full text search in an article database and get a presentation of extracts of the most similar parts in decreasing order with the source mentioned on Harvard citation format with the title of the publication, its type and link to the publisher’s full text PDF and my library’s full text PDF, so that we rapidly can find more information about any part that interests us.

		To achieve this goal, I need:
\begin{enumerate}
  \item configuration of a web server on epsilon,
  \item setup of an open-source database on epsilon containing PDFs,
  \item a collection of 10 000 PDFs manually downloaded from other databases (Or downloaded automatically by web crawler),
  \item a website that include a search bar and can send the user query to the search engine and receive the result then display it,
  \item a method to compare every keyword combinations in each sentence of an article with the database,
  \item rank the results of the comparison, put the result of the rank in decreasing order and displaying it on the new web page,
  \item design the search and display web page, make sure it is well decorated, so it can be user friednly,
  \item classify the searching results and generate a diagram on the result page, so it can be viewed conveniently,
  \item APIs defining interfaces in the system ,
  \item UML diagrams of the system,
  \item documentation of the code, 
\end{enumerate}


	\section{Background}
	\label{Background}
	%%%%%%%%%%%%%%%%%%%%%%%%%%%%%%%%%%%%%%%%%%%%%%%%%%%%%%%%%%%%%%%%%%%
% Background
% Year:
% 2016、2017
% Team:
% Wolverine、RCPL
% Members: 
% Dexter Chen(Wolverine/2016), Eric Chang(Wolverine/2016), Eric Lee(Wolverine/2016), 
% Jacky Wu(Wolverine/2016), Karthick Mani(Wolverine/2016), Kenvin Lo(Wolverine/2016), 
% Yu-cheng Chen(Wolverine/2016), Paul Lin(RCPL/2017)
% (Format:Name(Team/Year))
% Relative files:
% Main.tex, Background_Information_retrieval_on_existing_database.tex, Library.bib, Wolverine_Background_Chart_1.png
% Note: 
% Do not compile this file compile Main.tex to get the pdf file instead.
%%%%%%%%%%%%%%%%%%%%%%%%%%%%%%%%%%%%%%%%%%%%%%%%%%%%%%%%%%%%%%%%%%%
	
\subsection{Information retrieval on existing database}
	We live in the time when technology develops rapidly. Information grows in an exponential rate. Tague1981 forecasted the further into the future we go, the fewer the additional number of first-rate publications. The information develops from linear growth to exponential growth. Thus we can't rely on the old ways to find the information we need. We need new information retrieval methods to handle the big amount of data systematically. Howeverm, most of the information retrieval methods such as search engine can not search everything on the web. 
	Grehan2002 claimed a search engine which can only search the subset of the web it has ‘captured’ and included in its own database. Thus, we need to create a database to store these data and automatically update them.
	There are several online libraries currently available for us to get the academic articles or periodicals we need.
	Besides, they can be roughly divided into three groups according to the way they store articles based on the division used by National Taiwan University Library.

\paragraph{Index libraries}
	These kind of libraries store the index and abstract of the articles.
	They don't provide the full-text documents directly, but they may give the linkage to the publisher websites of articles.
	Besides, they can be categorized by the type of articles they include.
	
	\begin{itemize}		
		\item\textbf{Comprehensive topics}\\Libraries such as Web of Science, Scopus, Google Scholar...
		\item\textbf{Specialized topics}\\Libraries such as Compendex, BIOSIS Previews, PubMed, MEDline...		
	\end{itemize}
	
\paragraph{Publisher libraries}
	These libraries are created by the publishers themselves, so they provide the newest and complete the documents directly.
	Besides, they can also be categorize by the type of articles they include.
	
	\begin{itemize}		
		\item\textbf{Comprehensive topics}\\Libraries such as Science Direct, Springer Link, Wiley Online Library...
		\item\textbf{Specialized topics}\\Libraries such as Nature.com, Emerald Management Xtra, IEEE Xplore...	
	\end{itemize}
	
\paragraph{Aggregator libraries}
	These libraries do not publish the articles by themselves, but they still sometimes provide the user with the full-text articles.
	The way they do this is to negotiate with some of the publisher libraries and get the authorization of the articles.
	Libraries such as EBSCOhost, ProQuest, JSTOR...

	The comparison between these libraries can be found on Figure \ref{WBC1}.
	On the next section we will discuss about more details about some of the existing libraries.

\begin{figure*}[htb]
	\begin{center}
		\includegraphics[width=0.8\textwidth]{Wolverine_Background_Chart_1}
	\end{center}
	\caption{Comparison between three types of libraries.\label{WBC1}}
\end{figure*}
\newpage

\subsubsection{Introduction to libraries }

\begin{enumerate}
	
	\item\textbf{PubMed}
	\setlength{\parindent}{1em}
	 PMC (PubMed Central) was launched in 2000.
	 PubMed citations often include links to the full-text article on the publishers' web sites or in PMC and the Bookshelf.
	 PubMed is a free library which is used for searching reference papers and abstracts related to the biomedical topics.
	 The largest subset of PubMed is MEDLINE, which is a bibliographic database containing life sciences and biomedical information.
	 Both of them are built by National Library of Medicine. You may limit your search to MEDLINE only in PubMed.
	 A strong feature of PubMed is its ability to link MeSH(Medical Subject Headings) terms automatically. 
	 It is useful for people who want to find the medical articles. 
     Simple searches on PubMed can be carried out by entering the key words of a subject into PubMed's search window.
     PubMed translates this initial search formulation and automatically adds field names.
	 Like several libraries, people can find the specific result they desire by adding relevant MeSH terms, synonyms and Boolean operators.	 
	 The design philosophy of PubMed is based on full-text XML files, which are readable by machines, humans and moreover technology independent.
	 PubMed is classified into Index libraries, which is the prime reason that it is not able to provide full text for some papers.
	 The type of database used by PubMed is Microsoft SQL server, which is a relational database to store all of the archives such as XML, images, and PDF files supplementary.
	
	\item\textbf{IEEE Xplore}
	\setlength{\parindent}{1em}	
	IEEE is an acronym for Institute of Electrical and Electronics Engineers, which is one of the leading standard organizations in the world. 
	Besides, it is one of the world's largest technical professional organization dedicated to advancing technology for the benefit of humanity. 
	There are more than 420,000 IEEE members in over 160 countries.
	And IEEE Xplore is a scholarly research library formerly known as IEEE/IET Electronic Library (IEL).
	The articles covered by IEEE Xplore are mainly from the IEEE and the Institution of Engineering and Technology(IET).
	More than 3.5-million full-text documents are in the field of electrical, engineering, computer science, and electronics are provided in this library. 
	There are many features in IEEE. It can rank the articles according to their click through rates or download times. 
    If some articles are updated by an author, those who set research alert on it will receive a notification through email by IEEE.
    However, some of the features are available for members only.
    Many enterprises and schools are the members of IEEE.
    
	The front and user interface of IEEE library present the information on the screen, including the latest Angular, Jquery, HTML 5, CSS.
	Most of the HTML for PDF,  it is not only for journal (conference) articles but also for standards get dynamic transformations in real time and served through MarkLogic.
	Endeca, which is an Oracle product powers Xplore searches, is used in the search layer.
	All PDF files are fed through Endeca system.
	Endeca servers will provide the matching documents and Xplore platform will present it on the screen to the user.
	Beside, all contents are stored in oracle metadata which will be consumed by Endeca, MarkLogic Authentication, and Authorization services.
	
	\item\textbf{EBSCOhost}
	\setlength{\parindent}{1em}

	EBSCOhost is a popular reference which authorizes users to gain a great many full-text articles from proprietary databases.
	EBSCO Information Services, headquartered in Ipswich, Massachusetts, which is a division of EBSCO Industries Inc., the third largest private company in Birmingham, Alabama with annual sales of nearly $2$ billion according to the BBJ's 2013 Book of Lists.
    EBSCO offers library resources to customers in academic, medical, K–12, public library, law, corporate, and government markets. 
	Its products include EBSCONET, a complete e-resource management system, and EBSCOhost, which supplies a fee-based online research service with 375 full-text databases, a collectionof 600,000-plus ebooks, subject indexes, point-of-care medical references, and an array of historical digital archives.

    In 2010, EBSCO introduced its EBSCO Discovery Service (EDS) to institutions, which allows people to search a portfolio of journals and magazines

	\item\textbf{Google Scholar}
	\setlength{\parindent}{1em}
	
	According to , Google Scholar is a freely accessible web search engine that indexes the full text or metadata of scholarly literature across an array of publishing formats and disciplines. Released in beta in November 2004, the Google Scholar index includes most peer-reviewed online academic journals and books, conference papers, theses and dissertations, preprints, abstracts, technical reports, and other scholarly literature, including court opinions and patents. While Google does not publish the size of Google Scholar's database, third-party researchers estimated it to contain roughly 160 million documents as of May 2014 and an earlier statistical estimate published in PLOS ONE using a Mark and recapture method estimated approximately $80-90$ coverage of all articles published in English with an estimate of 100 million. This estimate also determined how many documents were freely available on the web.
	
	Google Scholar is similar in function to the freely available CiteSeerX and getCITED. It also resembles the subscription-based tools, Elsevier's Scopus and Thomson Reuters' Web of Science.
	
	Google Scholar allows users to search for digital or physical copies of articles, whether online or in libraries. "Scholarly" searches will appear using the references from "full-text journal articles, technical reports, preprints, theses, books, and other documents, including selected Web pages that are deemed to be "scholarly." Because most of Google Scholar's search results link directly to commercial journal articles, a majority of the time users will only be able to access a brief summary of the articles topics, as well as small amounts of important information regarding the article, and possibly have to pay a fee to access the entire article. The most relevant results for the searched keywords will be listed first, in order of the author's ranking, the number of references that are linked to it and their relevance to other scholarly literature, and the ranking of the publication that the journal appears in.
	
	Using its "group of" feature, it shows the available links to journal articles. In the 2005 version, this feature provided a link to both subscription-access versions of an article and to free full-text versions of articles; for most of 2006, it provided links to only the publishers' versions. Since December 2006, it has provided links to both published versions and major open access repositories, but it still does not cover those posted on individual faculty web pages; access to such self-archived non-subscription versions is now provided by a link to Google, where one can find such open access articles.
	
	Through its "cited by" feature, Google Scholar provides access to abstracts of articles that have cited the article being viewed. It is this feature in particular that provides the citation indexing previously only found in CiteSeer, Scopus and Web of Science. Through its "Related articles" feature, Google Scholar presents a list of closely related articles, ranked primarily by how similar these articles are to the original result, but also taking into account the relevance of each paper.
	
	At December 2009, Google Scholar is not yet available to the Google AJAX API.\\
	
	Google Scholar's legal database of US cases is extensive. Users can search and read published opinions of US state appellate and supreme court cases since 1950, US federal district, appellate, tax and bankruptcy courts since 1923 and US Supreme Court cases since 1791. Google Scholar embeds clickable citation links within the case and the How Cited tab allows lawyers to research prior case law and the subsequent citations to the court decision.The Google Scholar Legal Content Star Paginator extension inserts Westlaw and LexisNexis style page numbers in line with the text of the case.\\
	
	While most academic databases and search engines allow users to select one factor (e.g. relevance, citation counts, or publication date) to rank results, Google Scholar ranks results with a combined ranking algorithm in a "way researchers do, weighing the full text of each article, the author, the publication in which the article appears, and how often the piece has been cited in other scholarly literature". Research has shown that Google Scholar puts high weight especially on citation counts and words included in a document's title. As a consequence, the first search results are often highly cited articles.\\
		
	Limitations and criticism
	Quality — Some searchers consider Google Scholar of comparable quality and utility to commercial databases.The reviews recognize that its "cited by" feature in particular poses serious competition to Scopus and Web of Science. An early study, from 2007, limited to the biomedical field, found citation information in Google Scholar to be "sometimes inadequate, and less often updated". The coverage of Google Scholar may vary by discipline compared to other general databases.
	
	Coverage — Especially early on, some publishers did not allow Scholar to crawl their journals. Elsevier journals have been included since mid-2007, when Elsevier began to make most of its ScienceDirect content available to Google Scholar and Google's web search. As of February 2008 the absentees still included the most recent years of the American Chemical Society journals. Google Scholar does not publish a list of scientific journals crawled, and the frequency of its updates is unknown. It is therefore impossible to know how current or exhaustive searches are in Google Scholar, although a recent study estimates that Google Scholar can find almost $90$ (approximately 100 million) of all scholarly documents on the Web written in English. Nonetheless, it allows easy access to published articles without the difficulties encountered in some of the most expensive commercial databases.
	
	Matthew effect — Google Scholar puts high weight on citation counts in its ranking algorithm and therefore is being criticised for strengthening the Matthew effect; as highly cited papers appear in top positions they gain more citations while new papers hardly appear in top positions and therefore get less attention by the users of Google Scholar and hence fewer citations.
	
	Google Scholar effect – It is a phenomenon when some researchers pick and cite works appearing in the top results on Google Scholar regardless of their contribution to the citing publication because they automatically assume these works’ credibility and believe that editors, reviewers, and readers expect to see these citations.
	
	Incorrect field detection — Google Scholar has problems identifying publications on the arXiv preprint server correctly. Interpunctuation characters in titles produce wrong search results, and authors are assigned to wrong papers, which leads to erroneous additional search results. Some search results are even given without any comprehensible reason.
	
	Vulnerability to spam — Google Scholar is vulnerable to spam. Researchers from the University of California, Berkeley and Otto-von-Guericke University Magdeburg demonstrated that citation counts on Google Scholar can be manipulated and complete non-sense articles created with SCIgen were indexed from Google Scholar. They concluded that citation counts from Google Scholar should only be used with care especially when used to calculate performance metrics such as the h-index or impact factor. Google Scholar started computing an h-index in 2012 with the advent of individual Scholar pages. Several downstream packages like Harzing's Publish or Perish also use its data. The practicality of manipulating h-index calculators by spoofing Google Scholar was demonstrated in 2010 by Cyril Labbe from Joseph Fourier University, who managed to rank "Ike Antkare" ahead of Albert Einstein by means of a large set of SCIgen-produced documents citing each other (effectively an academic link farm).
	
	Inability to shepardize case law — As of 2010, Google Scholar was not able to shepardize case law, as Lexis can.
	
	Lack of screening for quality — Google Scholar strives to include as many journals as possible, including predatory journals, which "have polluted the global scientific record with pseudo-science, a record that Google Scholar dutifully and perhaps blindly includes in its central index."\\
	
	\item\textbf{Comparison Xplore}
	\setlength{\parindent}{1em}
	
	PubMed is a free library which contains many databases, like Medline, PreMedline and Publisher Supplied Citations.
    Medline is the largest subset of PubMed.
    One can also access MEDLINE through EBSCOhost.  
    EBSCOhost promises a large number of databases. 
	Many of them, such as MEDLINE and EconLit, are licensed from web content vendors.
    Others, such as Criminal Justice Abstracts, MasterFILE, are compiled by EBSCO itself.

    However, in contrast to other two libraries, the advanced search of PubMed is weak. 
    It does not show citation times or further information.
    And the only database which does not provide full-text documents is PubMed. 

    The documents in PubMed are almost related to the biomedical topics.
    IEEE contains more than one third documents in the field of electrical, engineering, computer science and electronics.
    And the articles in EBSCOhost comprises many fields, like business, education, laws, medical, computer science, and so on.

    Each library has its own features. PubMed can link the MeSH. EBSCOhost can be searched for the videos or photos concerned with the key words you enter. 
    IEEE has many features the other two databases don’t have, like “top downloads list”, “top search terms” and “custom setting” mentioned above.

    

\end{enumerate}

\todo[inline]{You need to add at least one library example of each library type, since you started with different types. Otherwise you could have focused on one type and motivated the focus. For the libraries containing articles you should add several since you are building one. You should also discuss the database techniques, including alternative ones to the used ones, such as MongoDB, Hadoop, etc. Try to compare features.}

	\section{Methods}
	\label{Methods}
	%%%%%%%%%%%%%%%%%%%%%%%%%%%%%%%%%%%%%%%%%%%%%%%%%%%%%%%%%%%%%%%%%%%
% Method
% Team:
% Wolverine
% Members: 
% Eric Lee, Jacky Wu, Karthick Mani, 
% Eric Chang, Dexter Chen, Peter Chen
% Relative files:
% Method_Wolverine.tex, Library.bib, WolverineChart.png
% Note:    
% Do not compile this file compile Main.tex to get the pdf file instead.
%%%%%%%%%%%%%%%%%%%%%%%%%%%%%%%%%%%%%%%%%%%%%%%%%%%%%%%%%%%%%%%%%%%
	
\subsection{Built a database containing ten thousand articles}
%\textit{\footnotesize Author:Dexter Chen, Eric Chang, Eric Lee, Jacky Wu, Karthick Mani, Kenvin Lo, Yu-cheng Chen.}\\

The most important thing on this subject is to build a database which contains 10,000 articles. To reach that target, this study will go to discuss the advantages and disadvantages of different databases and will decide the most suitable one to be the database of this study. At a meantime, this study will focus on how to use web crawlers to download articles automatically, which will contain in the database of this study

%Aim of this article is to give a brief idea of a database which is about to develop for the purpose of downloading the articles from open access databases using web crawler or web robot. And setting up a system that can allow the users to access the data in our database. we like to separate the article into 6 subsection based on their features, fallowed by then suggestions based on each feature. 


\subsubsection{Database Management Systems}

The relational database model was proposed by Edgar Codd in 1970, but because of the technological requirements it was not universal at that time. It was until 1980s that the first commercial relational database management systems began to appear.

A database management system (DBMS) is a computer software application that interacts with the user, other applications, and the database itself to capture and analyze data. Well-known DBMSs include MySQL, PostgreSQL, Microsoft SQL Server, Oracle, Sybase and IBM DB2. And they can support different kinds of databases.

%%%%%%%%%%%%%%%%% these are duplicated to the information following
%For building up such a database. We need a database management system (DBMS), a computer application that interacts with the user and other applications, captures and analyze data itself. Well-known DBMSs include MySQL, PostgreSQL, Microsoft SQL Server, Oracle, Sybase and IBM DB2. All of them can support different kinds of databases. This study includes numerous application and usage of such database as follows.
%MySQL is the second rank relational database management system (RDBMS) which is open-source. LAMP is an archetypal model of web service solution stacks, and its central component is MySQL. Web-based applications such as TYPO3 and MODx often use MySQL. MySQL is also applied in some famous website like Google, Facebook, and YouTube. It is able to be developed by the visual database design tool, MySQL Workbench.
%Same as Relational database, object-oriented database management systems has developed since the 1970s. Db4o which was launched in 2004 represents an object-oriented database model. It provides an easy interface to work with object-oriented programming languages and it also includes various object-oriented programming languages. For this reason, the programmer can work in one environment persistently.
%Finally, we'd like to introduce a DBMS, Neo4j. Neo4j is a graph database management system. It's one of the most popular GDBMS and ranks 20 in the popularity of DBMS. Unlike other databases, relationships take first priority in graph databases. Also, the model is simpler and expressive than those of relational databases, such as NOSQL databases. Neo4j is widely used by organizations and has 1,000,000+ downloads. Because Neo4j is easy to learn and use, it is more easier for beginners to get used to the structure of graph databases. This study also introduced three kinds of DBMS, and the following paragraph contains brief information about each kind of DBMS.

\begin{enumerate}
	\item\textbf{Object-oriented database}
	\setlength{\parindent}{1em}
	
	%An object-oriented database (OODBMS) is one of its kind, a database management system.\cite{WiKiauthor2013} The information in the database is represented as objects as used in object-oriented programming.
	
	An object database (also object-oriented database management system - OODBMS) is a database management system in which information is represented in the form of objects as used in object-oriented programming. Object databases are different from relational databases which are table-oriented.
	
	Because of tighter integration with the object-oriented language, the program is easier to maintain consistency with the same representation in both OODBMS and programming language.
	
	Although relational databases which are table-oriented might be similar to object-oriented databases, but they are actually different. The object-oriented database supports objects, classes, and inheritance in the database schema and query language.
	There are many advantages for OODBMS compared to the relational database management system (RDBMS) such as the performance, flexibility, and development cost.
	
	And OODBMS also have some disadvantages, the have mention 3 disadvantages for OODBMS. First, because the usage is forced to be similar to an object-oriented language. This makes maintaining and evolving is difficult. Second, the technique for store complex type of information takes additional computational resources. Third, the absence of a standard data model leads to design errors and inconsistencies.
	
	%And OODBMS also have some disadvantages, the \cite{Systems2010} have mention 3 disadvantages for OODBMS. First, because the usage is forced to be similar to an object-oriented language. This makes maintaining and evolving is difficult. Second, the technique for store complex type of information takes additional computational resources. Third, the absence of a standard data model leads to design errors and inconsistencies.
	
	\item\textbf{Relational database}
	\setlength{\parindent}{1em}
	
	A relational database is the most popular database used in the world. They can organize data into one or more tables of columns and rows, with a unique key identifying each row. Rows are also called records or tuples. Generally, each table represents one "entity type" (such as customer or product). The rows represent instances of that type of entity (such as "Lee" or "iPhone 6") and the columns representing values attributed to that instance (such as address or price).
	
	Considering the method of the organization of data, the relational database is much easier to understand and is flexible to manipulate the data. Besides SQL is easy in the relational database approach. For data organized in other structure, the query language either becomes complex or extremely limited in its capabilities. However, once the attributes of data become more and more, you'll need a large amount of tables to store your information. Therefore, the performance of relational database will decrease obviously.

	
	\item\textbf{Graph database}
	\setlength{\parindent}{1em}
	
	A graphical database uses graph structures for semantic queries with nodes, edges, and properties to represent and store data. Most of them are NoSQL in nature and store data in a key-value store or document-oriented database. Graph databases are a powerful tool for graph-like queries, for example computing the shortest path between two nodes in the graph.
	
	Graph database has several advantages to relational database. A graph database is often faster for associative data set and map more directly to the structure of object-oriented applications. They can scale more naturally to large data sets as they do not typically require expensive join operations. As they depend less on a rigid schema, they are more suitable to manage ad hoc and changing data with an evolving schema.
	
	On the other hand, graph database also comes with some disadvantages. For example, relational database is typically faster at performing the same operation on large numbers of data elements.\\
	
	According to database sturcture, there are several methods to store data. The two main directions are storing inside the database and storing out of the database.//
	
	We suggest not to store binary data in database if it is large. It will cause significant performance decrease and additional storage space. In contrast, we suggest to store binary data in the file system, and record the path in the database. It will not cause performance decrease when large binary data store into database, but the binary data can not automatically distribute with the database. Due to the PDF file will cost some performance issues even though it is small in size and our system has no requirement to automatic distribution. We suggest sorting the PDF file in the file system too.
	
\end{enumerate}

%\subsubsection{Database management system}
%The key feature which binds the relationship between user and administrator. Administrator's point of view, we like to give the best product to the end user. On the other hand, users strive to have convenient searching engine to find what they want at ease. For this purpose, I'm pressure to show two suggestions for enhancing the relationship between users and developer. First, setting up a word-ranking system. When user search for something with specific keyword, such as stem cell in medical area. In this moment, the word-ranking system will help the user by giving some suggested keywords. Of course, the word-ranking systems are developed based on users' searches and expert's suggestions to change the key words. This makes the system trust worthy.
 
%Secondly, building up a space in the database and allowing the user to change /edit the space based on their preference. The idea is referenced from well-known database, Wikipedia. It's so called "personalized searching".  By doing this, it would help user idealize their views on the system to what they want and suggest administrator the service which clients really want. It would lead to a win-win situation. 
	
	
\subsubsection{Web Crawler}
The web crawler is a program that can automatically browse through web pages, find out the information we assigned and store them. It has ability to process the data quickly and accurate to update a very large amount of data which are constantly being updated according to \cite{Liu2012}. It starts with a list of URL to visit, called the seeds. As crawler visits these URL, it identifies all the informations that we want, such as hyper links in the page and adds them to the list of URL to visit, called the crawl frontier. URL from the frontier is recursively visited according to a set of policies. If the crawler is performing archiving of websites, it copies and saves the information as it goes. The archives are usually stored in such a way they can be viewed, read and navigated as they were on the live web, but are preserved as 'snapshots' from \cite{Du2013}. We need to build up a web crawler to automatically visit a list of web page. Then find out which link in the page is valuable to download into our database.
	
	
%\subsubsection*{User account}
%One of the the main issue is how to create a user account that can connect between user and database. But the more important thing is to make sure database will not collapse by user who is not allowed to access to core part of database. 
 %To protect the database system security and privilege, this study introduces two methods for user account, principle of least privilege and role-based access control respectively. The principle of least privilege, also known as the principle of minimal privilege, means giving a user account only those those privileges which are essential to that user's work \cite{PrincipleLeastPrivilege}. The role-based access control is a policy neutral access control mechanism defined around privileges and roles. It can implement discretionary access control (DAC) or mandatory access control (MAC). The role-based access control is very easy to do user assignments as the components of this policy, such as role-permissions, etc. That is why it sometimes referred to as role-based security \cite{RoleBasedAccessControl}. The information and resources would not in danger due to these two methods will filter user depend on their authority and only allow the legitimate user to access. 
	
	
%\subsubsection*{User Interface}
%Understanding the types of visualizations people create by themselves for personal use. As part of this recent direction, we have studied a large collection of whiteboards in a research institution, where people make active use of combinations of words, diagrams and various types of visuals to help them further their thought processes. Our goal is to arrive at a better understanding of the nature of visuals that are created spontaneously during brainstorming, thinking, communicating, and general problem solving on whiteboards.\cite{Blascheck2016} We use the qualitative approaches of open coding, interviewing, and affinity diagramming to explore the use of recognizable and novel visuals, and the interplay between visualization and diagrammatic elements with words, numbers and labels. We discuss the potential implications of our findings on in- formation visualization design. Combining the advantage of visual thinking new standard of data processing, that visual nature of computers can challenge the first generation of hackers, An icon is an image, picture, or symbol representing a concept.\cite{Szpunar2010}
	
	
%\subsubsection*{Data Storage and Search Methods}
%The organization of data inside a database management system(DBMS) and retrieval methods is based on the database storage structure such as tables and indexes. There are several types of database storage structure such as XML, a textual data format. This advantage is self-describing and flexible in organizing data.\cite{ISI:000253400700005}Several considerations of data storage include right space allocation techniques, data compression techniques (if necessary), security and encryption and the access path to retrieve the data. Therefore, DBMS software will provide some method to optimize and minimum storage space of a database.
	
%\begin{figure*}[h]
%	\begin{center}
%		\includegraphics[scale=0.4]{WolverineChart}
%	\end{center}
%	\caption{Structure of our system}
%	\begin{center}
%		\includegraphics[scale=0.3]{WolverineChart2}
%	\end{center}
%	\caption{Database structure modes}
%\end{figure*}

\newpage % Ends the current page and causes all figures and tables to be printed

	%%%%%%%%%%%%%%%%%%%%%%%%%%%%%%%%%%%%%%%%%%%%%%%%%%%%%%%%%%%%%%%%%%%%%%%%%%%%%%%%%%%
% Team:
% EagleUnit
% Members: 
% Chinweze Ubadigha, Feng-Chun Hsia, Henry Peng, I-Chieh Lin, Jones Hou, Piyarul Hoque, Ray Chang
% Relative files:
% Main.tex, Background_EagleUnit.tex, Library.bib, EagleUnit_Background_Chart_1.png
% Note:
% Do not compile this file compile Main.tex to get the pdf file instead.
%%%%%%%%%%%%%%%%%%%%%%%%%%%%%%%%%%%%%%%%%%%%%%%%%%%%%%%%%%%%%%%%%%%%%%%%%%%%%%%%%%%
\subsection{Webpage Construction}
\textit{\footnotesize Author : Chinweze Ubadigha, Feng-Chun Hsia, Henry Peng, I-Chieh Lin, Jones Hou, Piyarul Hoque, Ray Chang.}\\

We are going to construct a public, user friendly web site, which provide service of scientific article searching.
To achieve this goal, we seperate our responsibility into parts:\\
\begin{itemize}
	\item Build up the web server.
	The web server is built by Django. Django is web framework which can highly construct by Python.
	Consider the Django framework that PostgresSOL, Nginx, and Gunicorn are recommand to build as an database sever, web sever, 
	and Python WSGI HTTP Server, respectively.
	\item Build a web page with a search bar.
	\item Place a list on the webpage which showing the search results.
	\item Build a metadata schema which contains all information in a PDF and then edit it in an XML structure. 
	\item Build up the APIs. The search bar should send the search string and the result list to tje search system 
	then the system recieve the results of searching and display it.
\end{itemize}
This article will provide a review through the tools and methodologies we use to achieve each work.
\subsubsection{Web Server}
For any web page, a web server is necessary.
There are three web server system used in our project. 
They play the role as web page framework, reverse proxy server, 
and database management, respectively.
Consider the ease to use and learn, and the generality of Python language 
that we choose Django, a Python based web server system for our work.


	%%%%%%%%%%%%%%%%%%%%%%%%%%%%%%%%%%%%%%%%%%%%%%%%%%%%%%%%%%%%%%%%%%%%%%%%%%%%%%%%%%%
% Team: Rainy
% Members: Rain  Dickson  Sareddy
% Relative files:
% Note: Do not compile this file compile Main.tex to get the pdf file instead.
%%%%%%%%%%%%%%%%%%%%%%%%%%%%%%%%%%%%%%%%%%%%%%%%%%%%%%%%%%%%%%%%%%%%%%%%%%%%%%%%%%%
\subsection{LSA}
\subsection{Similarity Comparison }
\subsection{Metadata Extraction}

	%%%%%%%%%%%%%%%%%%%%%%%%%%%%%%%%%%%%%%%%%%%%%%%%%%%%%%%%%%%%%%%%%%%%
% Method
% Team:
% Union
% Members: 
% Bernie Huang, Jim Lan, Hoang Tan, Kenny Hsu, Rahul Aditya, Tan Phat, Wei
% Relative files:
% Method_Union.tex
% Note:    
% Do not compile this file compile Main.tex to get the pdf file instead.
%%%%%%%%%%%%%%%%%%%%%%%%%%%%%%%%%%%%%%%%%%%%%%%%%%%%%%%%%%%%%%%%%%%
	
\subsection*{Title extraction}

\subsubsection*{Author abstraction}
\subsubsection*{Operate}
\begin{itemize}
	\item distinguish: distinguish all the words
	\begin{center}
		\includegraphics[width=0.8\columnwidth]{Union_Background_Chart_2}
	\end{center}
	\item classification: Divided into four categories\\ 	
	(a) Word Count $>$ 15\\
	(b) 15 $>$ Word Count $>$ 10\\
	(c) 10 $>$ Word Count $>$5\\
	(d) 5 $>$Word Count	
	\begin{center}
		\includegraphics[width=0.8\columnwidth]{Union_Background_Chart_3}
	\end{center}
	\item search : Search Word repetition rate
	\item statistics: Count the repetition rate of words
	\item sort: Show the number of rankings
\end{itemize}

\subsubsection*{Abstract extraction}

\nwepage

	
	%\section{Results and discussion}
	%\label{Results and discussion}
    %%%%%%%%%%%%%%%%%%%%%%%%%%%%%%%%%%%%%%%%%%%%%%%%%%%%%%%%%%%%%%%%%%%%%%%%%%%%%%%%%%%%
% Team:
% EagleUnit
% Members: 
% Chinweze Ubadigha, Feng-Chun Hsia, Henry Peng, I-Chieh Lin, Jones Hou, Piyarul Hoque, Ray Chang
% Relative files:
% Main.tex, Background_EagleUnit.tex, Library.bib, EagleUnit_Background_Chart_1.png
% Note:
% Do not compile this file compile Main.tex to get the pdf file instead.
%%%%%%%%%%%%%%%%%%%%%%%%%%%%%%%%%%%%%%%%%%%%%%%%%%%%%%%%%%%%%%%%%%%%%%%%%%%%%%%%%%%
\subsection{Web Server Structure}
By combining Django, PostgreSQL, NGINX, and Gunicorn, a robust web server was build.
The Django provide a very convenient web page framework to modify the outlooks and functions. 
For the database system, Postgres provide a much more powerful function and management then the default SQLite in Django.
By using Gunicor as an interface to the Django, NGINX makes the server more  fast and secured on communication with the internet requests.

\subsection{NGINX Server}
The NGINX server provide a particular internet domain to our web page,
which allowed us to connect the web page by using URL,
with out entering any port number. Additionally,
the NGINX server is not only playing the role of reverse proxy server of our Django server,
it can direct the user to any other domain in the main sever as well.
It means we can have other single HTML files not including in the Django server, or,
even an other Django project, and all linking to the same NGINX server.
NGINX can lead the user to projects according their own requests. 

\subsection{Web Page}
The final design of the search setting page has a search bar and several filters.
It allows users to enter the keywords, and decide which part of articles should the search system search in.
To display the articles found, we didn't shows them in the same page with search settings in the end. 
Instead, a link to a new page displaying all the metadata of search results was created.

    %%%%%%%%%%%%%%%%%%%%%%%%%%%%%%%%%%%%%%%%%%%%%%%%%%%%%%%%%%%%%%%%%%%%%%%%%%%%%%%%%%%%
% Team:
% Union
% Members: 
% Bernie Huan, Jim Lan, Hoang Tan, Kenny Hsu, Rahul Aditya, Tan Phat, Wei
% Relative files:
% Main.tex, Background_Union.tex, Library.bib, Union_Background_Chart_1.png, Union_Background_Chart_2.png, Union_Background_Chart_3.png, Union_Background_Chart_semi.png, Union_Background_Chart_sup1.png, Union_Background_Chart_sup2.png, Union_Background_Chart_sup3.png, Union_Background_Chart_WSD.png
% Note:
% Do not compile this file compile Main.tex to get the pdf file instead.
%%%%%%%%%%%%%%%%%%%%%%%%%%%%%%%%%%%%%%%%%%%%%%%%%%%%%%%%%%%%%%%%%%%%%%%%%%%%%%%%%%%%

\subsection{Result}
\textit{\footnotesize Author:Bernie Huan, Jim Lan, Hoang Tan, Kenny Hsu, Rahul Aditya, Tan Phat, Wei.}\\




\subsection{Discussion}



	\section{Result conclusions and future work}
	\label{Result conclusions and future work}
    %%%%%%%%%%%%%%%%%%%%%%%%%%%%%%%%%%%%%%%%%%%%%%%%%%%%%%%%%%%%%%%%%%%%%%%%
% Result&Conclusion
% Year:
% 2017
% Team:
% RCPL, Rainy
% Members: 
% Dickson Lee(Rainy/2017), Rain Wu(Rainy/2017), Sareddy Reddy(Rainy/2017), 
% Lewis Hsu(RCPL/2017), Tam-Van Ngo(RCPL/2017), Paul Lin(RCPL/2017)
% [Format:Name(Team/Year)]
% Relative files:
% Main.tex, Result&Conclusion.tex
% Note:    
% Do not compile this file compile Main.tex to get the pdf file instead.
%%%%%%%%%%%%%%%%%%%%%%%%%%%%%%%%%%%%%%%%%%%%%%%%%%%%%%%%%%%%%%%%%%%%%%%%%

\subsection{Results}
We already have an interesting and hard working course. In the course, we build Django, PostgreSQL database ,and develop search engine, after that we combine them together. Then, we can get the search result based on the search query. In order to have a user-friendly web page, we add some CSS programming files into it! We extend our database to 89 articles by manually downloading them from the Internet. Finally, we can display the similarity percentage and build a good-looking result page.

In brief, we have a functional web page implemented with search bar that you can copy a bunch of texts and return the percentage of similarity between the texts and the articles in the database.

    %%%%%%%%%%%%%%%%%%%%%%%%%%%%%%%%%%%%%%%%%%%%%%%%%%%%%%%%%%%%%%%%%%%%%%%%
% To do list
% Year:
% 2017
% Team:
% RCPL, Rainy
% Members: 
% Dickson Lee(Rainy/2017), Rain Wu(Rainy/2017), Sareddy Reddy(Rainy/2017), 
% Lewis Hsu(RCPL/2017), Tam-Van Ngo(RCPL/2017), Paul Lin(RCPL/2017)
% [Format:Name(Team/Year)]
% Relative files:
% Main.tex, Future_work.tex
% Note:    
% Do not compile this file compile Main.tex to get the pdf file instead.
%%%%%%%%%%%%%%%%%%%%%%%%%%%%%%%%%%%%%%%%%%%%%%%%%%%%%%%%%%%%%%%%%%%%%%%%%

\subsection{Future Work}
So far, we have implemented the full-text search function on our web page. 
After clicking the search icon, it can perform the full-text search through the PDF files in the database and extracts of the most similar parts. 
Besides, it can also display the title and its relevance in descending order on the result web page.

However, we did not perform the search function through a PDF reader.
Instead, we provide a search box on a web page which the user cannot click on a highlighted text and thereby perform a full-text search through the PDF reader. 
Therefore, it would be great to build or find a PDF reader which the strings inside the text can be clicked and perform full-text search through the PDF files in the database as the next step of the future work.
 
The current search process took quite a bit of time to a very small size of testing samples(89 articles), the response time of the system would be much more slower when the size of database increases. 
Therefore, optimization of a database, search function and the web server is needed. 
This will enhance the user experience while searching.
 
We also hope that more searching methods can be considered and implemented on the system. 
Apart from the performance of the web service, it is also worth providing a hyperlink to the original source of PDF file and using Harvard citation format.
Last but not least, visualization of the result is needed. The website should provide charts about the distribution of similarity score of different parts of the article to the search query, also provide service to show the desired result by selecting the relevant boxes(ex. rank by publishing year, show results based on different types of articles etc).
    

	\appendix
	\section{XML metadata sturcture}
	\label{XML}
	\input{Appendix_XML_Metadata_structure}
	
	\section{Automatic creation of metadata}
	\label{metadata_creation}
	%%%%%%%%%%%%%%%%%%%%%%%%%%%%%%%%%%%%%%%%%%%%%%%%%%%%%%%%%%%%%%%%%%%%%%%%%%%%%%%%%%%
% Team:
% Union
% Members: 
% Bernie Huan, Jim Lan, Hoang Tan, Kenny Hsu, Rahul Aditya, Tan Phat, Wei
% Relative files:
% Main.tex, Appendix_Automatic_creation_of_metadata.tex, Library.bib, Union_Background_Chart_1.png, Union_Background_Chart_2.png, Union_Background_Chart_3.png, Union_Background_Chart_semi.png, Union_Background_Chart_sup1.png, Union_Background_Chart_sup2.png, Union_Background_Chart_sup3.png, Union_Background_Chart_WSD.png
% Note:
% Do not compile this file compile Main.tex to get the pdf file instead.
%%%%%%%%%%%%%%%%%%%%%%%%%%%%%%%%%%%%%%%%%%%%%%%%%%%%%%%%%%%%%%%%%%%%%%%%%%%%%%%%%%%

\subsection{Automatic creation of metadata}



We are producing a program that automatically generate and extract metadata with natural language processing. 
We also strive to generate XML files with metadata extracted. 
In the best scenario, we will even try to create a search engine together with other groups. 
Also, creating a sutiable interface and structure with some finctions for users is necessary. 
Following discussion is our literature review on natural language processing.

\begin{figure*}[ht]
	\begin{center}
		\includegraphics[width=1.8\columnwidth]{Union_Background_Chart_1}
	\end{center}
	\caption{The process of metadata creation.}
\end{figure*}

\subsubsection*{Language understanding}

Natural language understanding (NLU) is a subtopic of natural language processing in artificial intelligence that deals with machine reading comprehension, it's considered an AI-hard problem.

For a machine to understand language, it first has to develop a mental map of words, their meanings and interactions with other words. It needs to build a dictionary of words, and understand where they stand semantically and contextually, compared to other words in their dictionary. To achieve this, each word is mapped to a set of numbers in a high-dimensional space, which are called “word embeddings”. Similar words are close to each other in this number space, and dissimilar words are far apart. Some word embeddings encode mathematical properties such as addition and subtraction.

After the machine has learned word embeddings, the next problem to tackle is the ability to string words together appropriately in small, grammatically correct sentences which make sense. This is called language modeling. Language modeling is one part of quantifying how well the machine understands language.

For example, given a sentence “I am eating pasta for lunch.”, and a word “cars”, if the machine can tell you with high confidence whether or not the word is relevant to the sentence “cars” is related to this sentence with a probability 0.01 and I am pretty confident about it, then that indicates that the machine understands something about words and contexts.

An even simpler metric is to predict the next word in the sentence. Given a sentence, for each word in its dictionary the machine assigns a probability of the word’s likeliness to appear next in the sentence. For example: “I am eating (     ).” To fill in the blank, a good language model would likely give higher probabilities to all edibles like “pasta”, “apple”, or “chocolate”, and it would give lower probability to other words in the dictionary which are contextually irrelevant like “taxi”, “building”, or “music”.

When users search for a sentence, how does the program understand the certain inputs of text? We could build a natural language understanding (NLU) system, in which the system's rules for semantic interpretation are learnt automatically from training data, which uses a set of possible yes-no questions that can be applied to data items.
After that, it follows rules for selecting the best questions at any node on the basis of training data by using a method for pruning trees to prevent over-training.

\subsubsection*{Name Recognition}

If users search for the word "Turkey", the results could be a country or an animal. 
The meaning is totally different and definitely make users very confused if he or she is not very familiar with the word "Turkey". 

There are a lot of misunderstandings like this if users search some words which have multiply meanings. 
Sometimes, the results are fully unrelated and this situation is always annoying. 
That would be troublesome when we count frequency of certain words to rank them.

Therefore, it is significantly crucial for a program to totally understand what users want by name recognition in natural language processing, finally they can find out the results much quicker and will not be confused anymore.

The method to improve the problem above is "categorize the words based on different subjects, topic, or genres" by using online database and python program. 
Metadata is limited in digital libraries and web resources, try to enlarge them with meaningful, organized and desired categories \cite{Kules2006}.

Besides dealing with mutiple-meaning words, the most important part of name recognition is to recognize the special names and terms such as locations, people name, country, even company names and academic terms.
Therefore, it is better for search engine to know what user want and huge name corpus are necessary. 
Plus, this work also can assist previous work.

With above effort, users' exploration and overviews of information could be better supported. It will be very convenient to find the results we want and lower the possibility of misunderstandings if users are not very familiar with finding the appropriate result in specific fields.
\cite{TunThuraThet2010} Users don't need to filter the results which are ranked by browsing frequency popularity but can just obtain the information and relevance by clicking the specific categories and some reasonable choices.

Plus, creating some choices for users is also vital because this make the searching much more oragnized. 
For example, if there are a lot of subtitles such as abstract, introduction, method, or references in some standard research articles, try to make some choices so that the users can easily find out what they want. 
There are a lot of different standard articles in the world.
 Making a suitable choices if someone want to creat a personalzed search engine and interface. 

Also, users are able to choose multiply fields if the results include a lot of relevant fields. 
That's a big motivation for people to handle these problems. 

A lot of online services have done similar tasks before. 
Thus, creating and using an online databases or automated metadata creation are to be recommended. 
The reason is there are many advantages, including integrating with the other cloud services or scaling with what users need such as how to categorize the categories. 
It is beneficial for people who would like to create a convenient and personalized database or metadata.\\\\\\


\todo[inline]{A summary on list format can be motivated, but then each item need to be brief and you should not introduce anything new like UMLS in it.}

\subsubsection*{Factuality}

In the process of producing metadata, which should be the most precise information and representing the text, validity of such metadata must be checked. Therefore, tools for fact checks are developed based on linguistic techniques. 

The tool could detect facts and excludes authors' subjective opinions \cite{Agerri2014}. From the authors's perspective, the two main set of tools having such functions is TIMEBANK and FACTBANK. (yes, the authors used capitalized name)

TIMEBANK was first proposed in \cite{pustejovsky2003timebank}. 
The idea was based on that English language has different tenses which could be exploited as signals for fact check. 
An example below could help to clarify the ideas. Let's examine these sentences:

\begin{itemize}
	\item I will go to Chimei museum tomorrow.
	\item Chimei museum is near Tainan District.
	\item I was in UK in 2012.
\end{itemize}

The first sentence is simple future tense which implies something has never actually happened, the second sentence is simple present tense which can directly imply facts, and the last sentence is in simple past tense which is about something already happened (which is facts), but is no longer a fact right now, so such fact must be used with caution. 

The reason for introducing such tool is that even scientific research articles can be glittering with subjective comments, opinions or even assumption from authors \cite{schultze2000confessional}. In addition to TIMEBANK, many other tools can be another filter for fact extraction. \cite{Dave2003mining} Identify words, clauses and phrases that show emotional state of the authors. 

The choice in expression of facts could also be a helpful indicator to show whether authors are subjectively supporting a cause, an opinion and so on \cite{Wiebe2005}. Among these mentioned approaches, this paper highly favors creation a kind of thesaurus compiled of linguistic signaling for non-factually statements such as FACTBANK, which is built by \cite{Sauri2009}. 
Following example shows how subjective statements can be picked out.

\begin{itemize}
	\item Channelization would guarantee high flow velocity in rivers, flooding and consequent degradation of riparian community (1a).
	\item Funding agencies would be happy with big entrepreneurs, instead of small and medium enterprises (1b).
	\item Tolerance to dictatorship would has negative influences on anarchist movement (2a).
	\item Tolerance to dictatorship would doom anarchist movement (2b).
\end{itemize}

It is easy to find in statement (1a) is an absolute fact. 
Statement (1b) is however affected by emotional state of authors. 
After re-writing (1b) into: Funding agencies lend more money with lower interest rate to big entrepreneurs, instead of small 
and medium enterprises,sentence (1b) become a face-based statement. 
In another case, statement (2a) is a fact-based statement while in statement (2b), authors are stressing their dislike toward dictatorship.

Fact checks in language generation is a new field but many useful tools have been developed. Each of them has their own function and could complement each others. In the limit of this study, we are using both of TIMEBANK and FACTBANK together for fact check.


\subsection{API and search engine}
An API is short-hand of  application programming interface, which include components to define a process of communication and data processing on computer. 
It is necessary to create a search engine as we mentioned at the beginning of this report.
A good example of API is HAPI on Python. 
According to \cite{Kochanov201615}, HAPI is a library included in Python. 
As the library has a collection of defined process, it allow users and other parties (which could be robots or others users at other ends) to interact with each other.
Main functions of the HAPI have been including sending uploads, data filtration and other manipulation. 
The outcomes of such process can be formatted with variable forms. 
As described by \cite{Hedbrant20162206}, programming interface can include following systems:
\begin{itemize}
	\item SOAP: Simple Object Access Protocol
	\item REST: Representational State Transfer
	\item UDDI: Universal Description, Discovery, and Integration
	\item WSDL: Web Service Definition Language
\end{itemize}
	Other programing system can be included as well. 
	SOAP is among a field attracting many recent researches. 
	A good review about SOAP could be found from the work of \cite{Hsieh2009424}. 
	SOAP has important roles in field monitoring as handling a huge amount of data can be trouble some in many senses. 
	SOAP in conjunction with other programming techniques, such as distributed computing, are employed because it can solve 3 problems listed below:
	\begin{itemize}
		\item Is the system scalable: 
		There are systems that suitable for a fixed amount of data, but not functional when the need of expending data storage comes in place. 
		With such system, programmers will have to build another system. 
		That would be a waste of human resource, time and money.
		\item Is the system reliable: 
		In other words, can data be stored safely. 
		Many data obtained in field monitoring are expensive. 
		Therefore, it is important to make sure data will be always there whenever a need of retrieval arisen. 
		Also, the data may be retrieved, processed and formatted by many people involved in the system.
		A reliable system will not be broken down easily and foremost, data will be safely stored.
		\item Is the system accessible: 
		This matter can be put in two questions 1- When we need to use the data, is it hard to retrieve it from the system. 
		Do we need trained personal to do that job.
		And if we do, which level of training is necessary. 
		2- How many people can access the system. 
		
	\end{itemize}
	It is proven in many engineering projects, the application of SOAP can help overcome these problems, especially in term of increasing accessiblity.
	\subsubsection{Search engine}
	As mentioned above, creating a search engine is an example of API application. 
	In this section we feel the need to give an example of how to create search engine with API. 
	We found an inspirational example from the work of \cite{Noei2016135}. 
	Although the authors not just using SOAP, but also more advanced Object Access Protocol such as Java, it shows that OAP in general improve code and design. 
	Consequently, it saves time and effort in developing new program.
	The authors also offer EXAF (EXample Applications Finder) so that program developers can find other examples that fits their interest.
	
\newpage % Ends the current page and causes all figures and tables to be printed
	
	\section{Harvard citation}
	\label{Harvard}
	%%%%%%%%%%%%%%%%%%%%%%%%%%%%%%%%%%%%%%%%%%%%%%%%%%%%%%%%%%%%%%%%%%%
% Harvard Citation
% Year:
% 2017
% Team:
% RCPL
% Members: 
% Lewis Hsu, Paul Lin, Tam-Van Ngo
% Relative files:
% Appendix_Harvard_citation.tex
% Note:    
% Do not compile this file compile Main.tex to get the pdf file instead.
%%%%%%%%%%%%%%%%%%%%%%%%%%%%%%%%%%%%%%%%%%%%%%%%%%%%%%%%%%%%%%%%%%%

\subsection{Harvard Citation format}

“According to \cite{James2008}, Harvard citation format is also known as the APA style or, more colloquially, as the ‘name(date)’ system. This is because an author’s surname in the text is followed by the date of the publication in brackets, and entries in the reference list are listed alphabetically, starting with the name and the initials of the author(s) followed by the date of publication for each entry.

Structure: Last name, First initial. (Year published). Article Title. Journal, [on line] Volume(Issue), pages. Available at: URL [Accessed Day Mo. Year].

Example: See the references in this report

For further references see \href{http://www.citethisforme.com/harvard-referencing}{harvard-referencing}


		
	%%%%%%%%%%%%%%%%%%%%%%%%%%%%%%%%%%%%%%%%%%%%%%%%%%%%%%%%%%%%%%%%%%%
	% References
	% Using Mendeley Desktop with library.bib
	%%%%%%%%%%%%%%%%%%%%%%%%%%%%%%%%%%%%%%%%%%%%%%%%%%%%%%%%%%%%%%%%%%%		
	\bibliographystyle{agsm_nurl}
	\bibliography{library}	
	\clearpage 
\end{document}  

\subsection{Search Settings}
So far our search system can only provide "single query" searching, which means the users can only search with a sigle key word everytime.
If multiple words was entered in the search bar, it will be processed as a single word.

In nowadays, almost every existing sientific article database provide multi-query and Boolean searching function,
to make it more convenient for the user to enter all possible key words,
and faster to convert the result.
To provide a more efficient information retrieve service,
multi-query search seems to be a necessary function.

\subsection{Search Result}
Providing more detail, generating the bibliogrophy, and refine the search result are all common functions in existing scientific article searching services.
IN our system, the result pages will display the metadata of all articles found very clearly.
However, it doesn't allow the user to sort the articles as their will. Also,
there's no function to operate or futher shrink the range of searching within the result.

\subsection{NGINX Server}
The NGINX server provide a particular internet donmain to our web page,
which allowed us to conect the web page by using URL,
with out entering any port number. Additionally,
the NGINX server is not only playing the role of reverse proxy server of our Django server,
it can direct the user to any other donmain in the main sever as well.
It means we can have other single HTML files not including in the Django server, or,
even an other Django project, and all linking to the same NGINX server.
NGINX can lead the user to projects according their own requiests. 
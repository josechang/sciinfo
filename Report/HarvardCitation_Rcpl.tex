%%%%%%%%%%%%%%%%%%%%%%%%%%%%%%%%%%%%%%%%%%%%%%%%%%%%%%%%%%%%%%%%%%%%%%%%%%%%%%%%%%%
% Team: RCPL
% Members: 
% Relative files:
% Note: Do not compile this file compile Main.tex to get the pdf file instead.
%%%%%%%%%%%%%%%%%%%%%%%%%%%%%%%%%%%%%%%%%%%%%%%%%%%%%%%%%%%%%%%%%%%%%%%%%%%%%%%%%%%
\subsection{Harvard Citation fomart}

Harvard is a style of referencing, primarily used by university students, to cite information sources.
Two types of citations are included:\\

\begin{itemize}
	\item In-text citations are used when directly quoting or paraphrasing a source. They are located in the body of the work and contain a fragment of the full citation. 
	\item Reference Lists are located at the end of the work and display full citations for sources used in the assignment.
\end{itemize}
For further references see \href{http://www.citethisforme.com/harvard-referencing}{harvard-referencing}

\subsubsection{Harvard Reference List Citations for Journal Articles Found on a Database or on a Website}

When citing journal articles found on a database or through a website, include all of the components found in a citation of a print journal, but also include the medium ([online]), the website URL, and the date that the article was accessed.\\
Structure:\\
\begin{itemize}
	\item Last name, First initial. (Year published). Article Title. Journal, [online] Volume(Issue), pages. Available at: URL [Accessed Day Mo. Year].
\end{itemize}
Example:\\
\begin{itemize}
	\item LRaina, S. (2015). Establishing Correlation Between Genetics and Nonresponse. Journal of Postgraduate Medicine, [online] Volume 61(2), p. 148. Available at: http://www.proquest.com/products-services/ProQuest-Research-Library.html [Accessed 8 Apr. 2015].
\end{itemize}


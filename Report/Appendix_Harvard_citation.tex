%%%%%%%%%%%%%%%%%%%%%%%%%%%%%%%%%%%%%%%%%%%%%%%%%%%%%%%%%%%%%%%%%%%
% Harvard Citation
% Year:
% 2017
% Team:
% RCPL
% Members: 
% Lewis Hsu, Paul Lin, Tam-Van Ngo
% Relative files:
% Appendix_Harvard_citation.tex
% Note:    
% Do not compile this file compile Main.tex to get the pdf file instead.
%%%%%%%%%%%%%%%%%%%%%%%%%%%%%%%%%%%%%%%%%%%%%%%%%%%%%%%%%%%%%%%%%%%

\subsection{Harvard Citation format}

“According to \cite{James2008}, Harvard citation format is also known as the APA style or, more colloquially, as the ‘name(date)’ system. This is because an author’s surname in the text is followed by the date of the publication in brackets, and entries in the reference list are listed alphabetically, starting with the name and the initials of the author(s) followed by the date of publication for each entry. For example


\begin{itemize}
	\item Sharples, Michael (Ed.). Computer Supported Collaborative Writing. London:
Springer-Verlag, 1993.
	\item Speck, Bruce W., Teresa R. Johnson, Catherine Dice, and Leon B. Heaton.
Collaborative Writing: An Annotated Bibliography. Westport, Connecticut: Greenwood Press, 1999.
	\item Tang, Catherine. ‘Effects of collaborative learning on the quality of
assignments.’ Teaching and Learning in Higher Education. Eds. Barry
Dart and Gillian Boulton-Lewis. Pp. 103–23. Melbourne: Australian
Council for Educational Research, 1998.
	\item Tang, Catherine. ‘Effects of collaborative learning on the quality of
assignments.’ Teaching and Learning in Higher Education. Eds. Barry
Dart and Gillian Boulton-Lewis. Pp. 103–23. Melbourne: Australian
Council for Educational Research, 1998.
Zammuner, Victoria L. ‘Individual and co-operative computer writing
and revising: Who gets the best results?’ Learning and Instruction 5
(1995) 101–24.
\end{itemize}
\documentclass[a4paper]{article} % Document type

\ifx\pdfoutput\undefined
    %Use old Latex if PDFLatex does not work
   \usepackage[dvips]{graphicx}% To get graphics working
   \DeclareGraphicsExtensions{.eps} % Encapsulated PostScript
 \else
    %Use PDFLatex
   \usepackage[pdftex]{graphicx}% To get graphics working
   \DeclareGraphicsExtensions{.pdf,.jpg,.png,.mps} % Portable Document Format, Joint Photographic Experts Group, Portable Network Graphics, MetaPost
   \pdfcompresslevel=9
\fi

\usepackage{amsmath,amssymb}   % Contains mathematical symbols
\usepackage[ansinew]{inputenc} % Input encoding, identical to Windows 1252
\usepackage[english]{babel}    % Language
\usepackage[round,authoryear]{natbib}  %Nice author (year) citations
%\usepackage[square,numbers]{natbib}     %Nice numbered citations
%\bibliographystyle{unsrtnat}           %Unsorted bibliography
\bibliographystyle{plainnat}            %Sorted bibliography

\addtolength{\topmargin}{-20mm}% Removes 30mm from the top margin
\addtolength{\textheight}{10mm}% Adds it to the text height


\begin{document}               % Begins the document

\title{Automatic creation of metadata/markup by use of natural language processing of full text articles}
\author{First name Last name \\ student number \\ email} 
%\date{2010-10-10}             % If you want to set the date yourself.

\maketitle                     % Generates the title

\section*{overview}
\label{sec:prob}

We�re producing a program which automatically generate metadata such as authors� name, date of publishing, name of articles� and more importantly auto-abstract for full text articles
We are interested in features for either more convenient use of the program or improving precise data generation
There are 5 related problems.\\
\includegraphics[scale=0.3]{../Picture1}\\
Figure 1. Overview of metadata creation process.\\

\section*{Problem 1}
\label{sec:prob}
When  users  search  with  a  sentence,  how  do  the  program  understand  the  certain  input  of  text?  

\section*{Method 1}
\label{sec:meth}

Building  a  natural  language  understanding  (NLU)  system.
\section*{Solution 1}
\label{sec:solu}

Use  a  set  of  possible  yes-no  questions  that  can  be  applied  to  data  items,  then  follow  a  rule  for selecting  the  best  question  at  any  node  on  the  basis  of training  data,  which  has  a  method  for  pruning trees to prevent over-training.


\section*{Problem 2}
\label{sec:prob}
When users search Turkey, the results could be a country or an animal. Sometimes, the results are totally unrelated. 

\section*{Method 2}
\label{sec:meth}
categorize the results based on different subjects or genres

\section*{Solution 2}
\label{sec:solu}

It�s significantly crucial for search engine to understand what users want by name recognition in natural language processing.  Digital libraries and web resources have limited metadata, augmenting them with meaningful, stable and desired categories. Information can enable better overviews and support user exploration.  

\section*{Problem 3}
\label{sec:prob}
Word sense disambiguation

\section*{Method 3}
\label{sec:meth}
Word sense disambiguation is an important step in natural language processing. This is the step where words with different meaning will be listed in different category (Abualhaija and Zimmermann, 2016). WSD has been done with three main approaches: supervised disambiguation (Abualhaija and Zimmermann, 2016), semi-supervised approach (Ben Aouicha et al., 2016), and more recently unsupervised approach (Yoon et al., 2006). Research for unsupervised approach has been developed quickly and application of this approach has been found in WSD for not-so-popular language such as Korean. 

\section*{Solution 3}
\label{sec:solu}
The project team has decided to pursuit the unsupervised approach. Implementation will be made in term of symnonym grouping (Navigli, 2009) and context clustering (Wang et al., 2009).

\section*{Problem 4}
\label{sec:prob}
The readers do not know what the connected words meaning

\section*{Method 4}
\label{sec:meth}
Compute and Analyze in natural language processing

\section*{Solution 4}
\label{sec:solu}
I suggest we should use the Python language to complete this task. 
It can easily conpute and analyze the words in the articles.
Phthon can compute and analyze by separating the connected words.
The readers can know what do the words mean when they are reading the articles .



\section*{Problem 5}
\label{sec:prob}
Natural language processing help us extract the important information from the full text article.
How could we make it more efficiently and precisely?


\section*{Method 5}
\label{sec:meth}
Query reduction to single sub-query


\section*{Solution 5 111}
\label{sec:solu}
 The performance of the machine is better in the short query rather than long query. Thus, it is an important issue to reduce the query to many sub-query.  The first is extracting the single sub-query by the existing features. Then, We combine these features to the reduction�s technique. We could find that it is more efficient than just analyze the original query. 



\section*{Reference}
\label{sec:refe}
[1].Jin 2008, Effectiveness Web Search Results for Genre and Sentiment Classification
\\\
[2].Bill 2006, Categorizing Web Search Results into Meaningful and Stable Categories Using Fast-Feature Techniques
\\\
[3].Weiscbedel 2006 White Paper on Natural Language Processing
\\\
[4].Collins 2011 Natural Language Processing Machine Learning Research
\\\
[5].Manish Gupta 2015, Information Retrieval with Verbose Queries, Foundations and Trends in Information Retrieval
\\\
[6].Julia Hirschberg 2015, Advances in natural language processing, Science
\\\
[7].Shapiro1982, A knowledge engineering approach to natural language understanding 
\\\
[8].Kuhn1995, The Application of Semantic Classification Trees to Natural Language Understanding
\\\
[9].Abualhaija, S.and Zimmermann, K.-H. 2016. D-Bees: A novel method inspired by bee colony optimization for solving word sense disambiguation. Swarm and Evolutionary Computation, 27, 188-195.
\\\
[10]. Ben Aouicha, M., Hadj Taieb, M. A.and Ezzeddine, M. 2016. Derivation of �is a� taxonomy from Wikipedia Category Graph. Engineering Applications of Artificial Intelligence, 50, 265-286.
\\\
[11]. Navigli, R. 2009. Word sense disambiguation: A survey. ACM Computing Surveys (CSUR), 41, 10. 
\\\
[12]. Wang, H., Missura, O., G�rtner, T.and Wrobel, S. 2009. Context-based clustering of image search results. In: KI 2009: Advances in Artificial Intelligence. Springer.
\\\
[13]. Yoon, Y., Seon, C.-N., Lee, S.and Seo, J. 2006. Unsupervised word sense disambiguation for Korean through the acyclic weighted digraph using corpus and dictionary. Information Processing and Management, 42, 710-722.
\\\


\end{document}      % End of the document

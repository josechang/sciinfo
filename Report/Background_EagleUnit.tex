\section*{XML metadata structure}
\label{sec:abs}


Our responsipility is to construct an XML metadata structure.


%%%%%%%%%%%%%%%%%%%%%%%%%%%%%%%%%%%%%%%%%%%%%%%%%%%%%%%%%%%%%%%%%%%%%%%%%%%%%%%%%%%
% 1. Different standards of metadata
%%%%%%%%%%%%%%%%%%%%%%%%%%%%%%%%%%%%%%%%%%%%%%%%%%%%%%%%%%%%%%%%%%%%%%%%%%%%%%%%%%%

\subsubsection{Different standards of metadata}
\label{sec:mets}
There's many different standards existing to describe the metadata in different fields and applications. We list 5 of them and give a brief introduction to each one.

\begin{enumerate}
	\item METS\\
	{\bf Introduction}\\
	
	{\bf Adventage}\\
	\begin{enumerate}
		
	\end{enumerate}	
	{\bf Disadventage}\\
	\begin{enumerate}
		
	\end{enumerate}
	{\bf Conclusion}\\	
	
	%%%%%%%%%%%%%%%%%%%%%%%%%%%%%%%%%%%%%%%%%%%%%%%%%%%%%%%%%%%%%%%%%%%%%%%%%%%%%%%%%%%
	\item MODS\\
	{\bf Introduction}\\
	
	{\bf Adventage}\\
	\begin{enumerate}
		
	\end{enumerate}	
	{\bf Disadventage}\\
	\begin{enumerate}
		
	\end{enumerate}
	{\bf Conclusion}\\
	
	%%%%%%%%%%%%%%%%%%%%%%%%%%%%%%%%%%%%%%%%%%%%%%%%%%%%%%%%%%%%%%%%%%%%%%%%%%%%%%%%%%%
	\item METS+MODS+PREMIS\\
	{\bf Introduction}\\
	
	{\bf Adventage}\\
	\begin{enumerate}
		
	\end{enumerate}	
	{\bf Disadventage}\\
	\begin{enumerate}
		
	\end{enumerate}
	{\bf Conclusion}\\
	
	%%%%%%%%%%%%%%%%%%%%%%%%%%%%%%%%%%%%%%%%%%%%%%%%%%%%%%%%%%%%%%%%%%%%%%%%%%%%%%%%%%%
	\item MARC 21\\
	{\bf Introduction}\\
	
	{\bf Adventage}\\
	\begin{enumerate}
		
	\end{enumerate}	
	{\bf Disadventage}\\
	\begin{enumerate}
		
	\end{enumerate}
	{\bf Conclusion}\\
	
	%%%%%%%%%%%%%%%%%%%%%%%%%%%%%%%%%%%%%%%%%%%%%%%%%%%%%%%%%%%%%%%%%%%%%%%%%%%%%%%%%%%	
	\item MARCXML\\
	{\bf Introduction}\\
	
	{\bf Adventage}\\
	\begin{enumerate}
		
	\end{enumerate}	
	{\bf Disadventage}\\
	\begin{enumerate}
		
	\end{enumerate}
	{\bf Conclusion}\\
	
	%%%%%%%%%%%%%%%%%%%%%%%%%%%%%%%%%%%%%%%%%%%%%%%%%%%%%%%%%%%%%%%%%%%%%%%%%%%%%%%%%%%
	\item Dublin Core\\
	{\bf Introduction}\\
	
	{\bf Adventage}\\
	\begin{enumerate}
		
	\end{enumerate}	
	{\bf Disadventage}\\
	\begin{enumerate}
		
	\end{enumerate}
	{\bf Conclusion}\\
	
	%%%%%%%%%%%%%%%%%%%%%%%%%%%%%%%%%%%%%%%%%%%%%%%%%%%%%%%%%%%%%%%%%%%%%%%%%%%%%%%%%%%
	\item IAFA/Whois++ Templates\\
	{\bf Introduction}\\
	
	{\bf Adventage}\\
	\begin{enumerate}
		
	\end{enumerate}	
	{\bf Disadventage}\\
	\begin{enumerate}
		
	\end{enumerate}
	{\bf Conclusion}\\	

\end{enumerate}

More detailed introduction could be found in {\bf\cite{1:1:1}} and {\bf\cite{Rachel:2009:reviewofmetadataformats}}.

%%%%%%%%%%%%%%%%%%%%%%%%%%%%%%%%%%%%%%%%%%%%%%%%%%%%%%%%%%%%%%%%%%%%%%%%%%%%%%%%%%%
% 2. Necessary elements of XML metadata
%%%%%%%%%%%%%%%%%%%%%%%%%%%%%%%%%%%%%%%%%%%%%%%%%%%%%%%%%%%%%%%%%%%%%%%%%%%%%%%%%%%

\subsubsection*{2. Necessary elements of XML metadata with DTD}
\label{sec:mets}
{\bf\cite{Ruey-Shun:2003:DevelopinganXMLframeworkformetadatasystem}} suggest that an XML metadata discribed according the DTD include three necessary elements:
\begin{enumerate}
	\item Structure\\
	The major execution ability of structure includes parser for well-formed XML and
	valid DTD structure, authoring tool for editing.
	
	\item Depth\\
	Basically, there are two sorts of fields: Fixed-length fields and variable fields.
	Fixed-length fields are general types and character-indication types Sub-field, whether
	fixed-length fields or variable fields, might contain both fixed-length fields and
	variable fields. According to the reason above, the process ability of the system has to
	cover the situation
	
	\item Scope\\
	The connections must involve simple object, time, space, people, and event. 
\end{enumerate}


%%%%%%%%%%%%%%%%%%%%%%%%%%%%%%%%%%%%%%%%%%%%%%%%%%%%%%%%%%%%%%%%%%%%%%%%%%%%%%%%%%%
% The bibliography
%%%%%%%%%%%%%%%%%%%%%%%%%%%%%%%%%%%%%%%%%%%%%%%%%%%%%%%%%%%%%%%%%%%%%%%%%%%%%%%%%%%
%\bibliography{Bibliography_template} %Read the bibliography from a separate file

\begin{thebibliography}{99}
\bibitem[Barker(2010)]{1:1:1}
Phil Barker.
\newblock \emph{Metadata for Learning Materials: an Overview of Existing Standards and Current Developments}.
\newblock Technology, Instruction, Cognition and Learning vol 7 (3-4) 2010
\newblock http://www.oldcitypublishing.com/TICL/TICLcontents/TICLv7n3-4contents.html


\bibitem[Rachel Heery.(2009)]{Rachel:2009:reviewofmetadataformats}
Rachel Heery.
\newblock \emph{Review of Metadata Formats}.
\newblock "Review of metadata formats", Program, Vol. 30 Iss 4 pp. 345 - 373,1996
\newblock http://dx.doi.org/10.1108/eb047236

\bibitem[Ruey-Shun Chen(2003)]{Ruey-Shun:2003:DevelopinganXMLframeworkformetadatasystem}
Ruey-Shun Chen.
\newblock \emph{Developing an XML framework for metadata system}.
\newblock ISICT '03 Proceedings of the 1st international symposium on Information and communication technologies
\newblock http://dl.acm.org/citation.cfm?id=963653

\bibitem[Sharon Cheslow(2014)]{Sharon Cheslow:2014:METSForTheCulturalHeritageCommunity}
Sharon Cheslow.
\newblock \emph{METS For The Cultural Heritage Community: A	Literature Review}.
\newblock Library Philosophy and Practice (e-journal). Paper 1162.
\newblock http://digitalcommons.unl.edu/libphilprac/1162/

\bibitem[Rebecca Guenther(2003)]{Rebecca Guenther:2003:NewMetadataStandardsforDigitalResources}
Rebecca Guenther.
\newblock \emph{New Metadata Standards for Digital Resources: MODS and METS}.
\newblock Portal: Libraries and the Academy, Johns Hopkins University Press (2003)
\newblock http://onlinelibrary.wiley.com/doi/10.1002/bult.268/pdf

\end{thebibliography}


%%%%%%%%%%%%%%%%%%%%%%%%%%%%%%%%%%%%%%%%%%%%%%%%%%%%%%%%%%%%%%%%%%%%%%%%%%%%%%%%%%%
% figures
%%%%%%%%%%%%%%%%%%%%%%%%%%%%%%%%%%%%%%%%%%%%%%%%%%%%%%%%%%%%%%%%%%%%%%%%%%%%%%%%%%%
\clearpage % Ends the current page and causes all figures and tables to be printed

\begin{figure*}[p] % The * makes the figure span both columns, p places the figure on a float page
	\begin{center}
		\includegraphics[scale=1.0]{hw3.pdf}
	\end{center}
	\caption{hierarchical overview of the methods/solutions}
	\label{fig:hw3}
\end{figure*}




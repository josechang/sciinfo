\lstdefinestyle{mystyle}{
	backgroundcolor=\color{green!20},   
	commentstyle=\color{blue},
	keywordstyle=\color{red},
	numberstyle=\tiny\color{gray}\noncopynumber,
	stringstyle=\color{purple},
	basicstyle=\ttfamily,
	breakatwhitespace=false,         
	breaklines=true,                 
	captionpos=b,                    
	keepspaces=true,  
   	numbers=none,    	           
	numbersep=4pt,                  
	showspaces=false,                
	showstringspaces=false,
	showtabs=false,       
	keepspaces=true,           
	tabsize=2
	columns=fullflexible,
	language={python},
	breakindent=0pt,
	frame=single,	
}



\newcommand{\noncopynumber}[1]{
	\BeginAccSupp{method=escape,ActualText={}}
	#1
	\EndAccSupp{}
}


\lstset{style=mystyle}
\begin{enumerate}
	\item Django installation
\begin{itemize}
	\item Step 1 :
	Open the terminal and create folder named mydjango
\begin{lstlisting}[language=python]
$ mkdir mydjango
$ cd mydjango

\end{lstlisting}
\item Step 2:
Create a vitual environment named mydjangovenv
\begin{lstlisting}[language=python]
$ virtualenv mydjangovenv
$ cd mydjangovenv
$ source bin/activate

\end{lstlisting}
\item Step 3:
Install django

\begin{lstlisting}[language=python]
$ python -m pip install django

\end{lstlisting}
\item Step 4:
Create a project named mysite and make it run in server

\begin{lstlisting}[language=python]
$ django-admin.py startproject mysite
$ python manage.py runserver 0.0.0.0:8000

\end{lstlisting}

\item Step 5:
Create apps named firstapp

\begin{lstlisting}[language=python]
$ python manage.py startapp firstapp

\end{lstlisting}
\end{itemize}
\end{enumerate}
% !TEX encoding = UTF-8 Unicode
\documentclass[a4paper,twocolumn,twoside]{article}
%%%%%%%%%%%%%%%%%%%%%%%%%%%%%%%%%%%%%%%%%%%%%%%%%%%%%%%%%%%%%%%%%%%
% Packages using
%%%%%%%%%%%%%%%%%%%%%%%%%%%%%%%%%%%%%%%%%%%%%%%%%%%%%%%%%%%%%%%%%%%
\ifx\pdfoutput\undefined
	\usepackage[dvips]{graphicx}
	\DeclareGraphicsExtensions{.eps}
\else
	\usepackage[pdftex]{graphicx}
	\DeclareGraphicsExtensions{.pdf,.jpg,.png,.mps}
	\pdfcompresslevel=9
\fi

\usepackage[utf8]{inputenc}
\usepackage[english]{babel}
\usepackage{indentfirst}
\addtolength{\topmargin}{-20mm}
\addtolength{\textheight}{10mm}
\usepackage{amsmath} % Added to get modern math environments
\usepackage{amssymb,amsfonts} %Added to get math
\usepackage{amsthm} % Added to get theorems
\usepackage{natbib} % Added to get better bibliography
\usepackage{soul} %underline
\usepackage{url}
%\usepackage{listings}
%\usepackage{color}
% Special hack below to break the URL!
\def\UrlBreaks{\do\.\do\@\do\\\do\/\do\!\do\_\do\|\do\;\do\>\do\]%
 \do\)\do\,\do\?\do\'\do+\do\=\do\#\do\i\do\m\do\t\do\a\do\x}%
\urlstyle{rm} %
\usepackage[bookmarks=true,bookmarksnumbered=true,hypertexnames=true,breaklinks=true,colorlinks=true]{hyperref}
\hypersetup{
pdfauthor = {Torbj\"{o}rn E. M. Nordling},
pdftitle = {Information Retrieval and Processing--Setup of a Full Text System Implementing Automatic Metadata Extraction and Visualization},
pdfsubject = {Technical Report},
pdfkeywords = {Information retrieval, Metadata}}

% Headers and footers (must be after the document settings)
\usepackage{fancyhdr} %Custom header package
\pagestyle{fancy} %Turn on fancy headers
\fancyhead{} %Clears default layout
\fancyfoot{} %Clears default layout
\fancyhead[LO,RE]{\small \normalfont \leftmark} %Adds section headline to header
\fancyhead[LE,RO]{\slshape \rightmark} %Adds subsection headline to header
\fancyfoot[LO,RE]{\href{http://www.nordlinglab.org/ScientificInformation}{nordlinglab.org/ScientificInformation}}
\fancyfoot[RO,LE]{Information Retrieval and Processing}
\fancyfoot[CO,CE]{\thepage}
\renewcommand{\headrulewidth}{0.4pt} %Header line
\renewcommand{\footrulewidth}{0.4pt} %Footer line

% Select what to do with todonotes: 
% \usepackage[disable]{todonotes} % notes not showed
\usepackage[draft]{todonotes}   % notes showed

%%%%%%%%%%%%%%%%%%%%%%%%%%%%%%%%%%%%%%%%%%%%%%%%%%%%%%%%%%%%%%%%%%%

\begin{document} 
	
	\title{Information Retrieval and Processing-RCPL report}
	\author{Lewis Hsu, Paul Lin, Tam Ngo}  % Please add your name here in alphabetic order (except Prof. N who is last)
	\maketitle   
	
	\section{Introduction}
	\label{Introduction}
	Note: I just have create the frame, you guys feel free to edit.\\
	This technical report contains information of what have done by RCPL team.
	
	
	\subsection{Aim}
	\label{aim}

	\todo[inline]{This section should be updated based on what you are building.}
	2017 user story:
	As a researcher, I need a PDF reader where I can click on highlighted text and thereby perform a full text search in an article database and get a presentation of extracts of the most similar parts in decreasing order 
	with the source mentioned on Harvard citation format with the title of the publication, its type and link to the  publisher’s full text PDF and my library’s full text PDF, so that I rapidly can find more information about any part that interests me.\\
	
	
		
	\section{Background}
	\label{Background}
    \subsection{Structure of the system}
	\subsection{Docker engine}
	\subsection{Nginx}
	\subsection{Django}
	\subsection{Postgres}

	\section{Methods}
	\label{Methods}
	

	
	\section{Results and discussion}


	\section{Conclusions and future work}

		
	%%%%%%%%%%%%%%%%%%%%%%%%%%%%%%%%%%%%%%%%%%%%%%%%%%%%%%%%%%%%%%%%%%%
	% References
	% Using Mendeley Desktop with library.bib
	%%%%%%%%%%%%%%%%%%%%%%%%%%%%%%%%%%%%%%%%%%%%%%%%%%%%%%%%%%%%%%%%%%%		
	\bibliographystyle{agsm_nurl}
	%\bibliography{library}	
	\clearpage 
\end{document}  

\documentclass[a4paper,twocolumn]{article} % Document type

\ifx\pdfoutput\undefined
    %Use old Latex if PDFLatex does not work
   \usepackage[dvips]{graphicx}% To get graphics working
   \DeclareGraphicsExtensions{.eps} % Encapsulated PostScript
 \else
    %Use PDFLatex
   \usepackage[pdftex]{graphicx}% To get graphics working
   \DeclareGraphicsExtensions{.pdf,.jpg,.png,.mps} % Portable Document Format, Joint Photographic Experts Group, Portable Network Graphics, MetaPost
   \pdfcompresslevel=9
\fi

\usepackage{amsmath,amssymb}   % Contains mathematical symbols
%\usepackage[ansinew]{inputenc} % Input encoding, identical to Windows 1252
\usepackage[english]{babel}    % Language
%\usepackage[round,authoryear]{natbib}  %Nice author (year) citations
\usepackage[square,numbers]{natbib}     %Nice numbered citations
%\bibliographystyle{unsrtnat}           %Unsorted bibliography
\bibliographystyle{unsrt}            %Sorted bibliography

\addtolength{\topmargin}{-30mm}% Removes 30mm from the top margin
\addtolength{\textheight}{30mm}% Adds it to the text height


\begin{document}               % Begins the document

\title{Sprint 3 Report of HankyPanky}
\author{Kevin Lo, R96031109, owl3808@gmail.com \\ Tim Hsia, N16044496, peace11130@yahoo.com.tw}
%\date{2010-10-10}             % If you want to set the date yourself.

\maketitle                     % Generates the title




%%%%%%%%%%%%%%%%%%%%%%%%%%%%%%%%%%%%%%%%%%%%%%%%%%%%%%%%%%%%%%%%%%%%%%%%%%%%%%%%%%%
% Instructions regarding the report
%%%%%%%%%%%%%%%%%%%%%%%%%%%%%%%%%%%%%%%%%%%%%%%%%%%%%%%%%%%%%%%%%%%%%%%%%%%%%%%%%%%

\section*{Background}
\label{sec:prob}

Nowadays Internet technology push lots of information be generated. These huge among of information is impossible for human to read each of them. So lots of information retrieval technology had been developed. Text categorization systems are one of them. It is useful in a wide variety of tasks, such as routing news and e-mails. In order to identifying junk email, or handling intelligence reports. There are many techniques to approach the goal, such as support vector machines, k-nearest neighbor algorithm, and neural networks.
  
According to (Gabrilovich and  Markovitch, 2007), one of the early method is bag of words (BOW). The features commonly used are the individual words appearing in the training documents, and the order of the words is ignored. The value of a feature for a particular document is usually its occurrence frequency. Although this representation scheme is easy and efficiency,  there are still some disadvantage. One of them is treating each document as a bag of the words, and is therefore known as the bag of words (BOW) approach (Salton and McGill, 1983).
  
To improve the accuracy, there have been a number of methods to add outside knowledge to effect machine learning techniques. Transfer learning approaches (Bennett et al., 2003; Do and Ng, 2005; Sutton and McCallum, 1998; Raina et al., 2006) employ information from some related learning tasks. Pseudo-relevance feedback (Ruthven and Lalmas, 2003) use information of several top-ranked documents. Semi-supervised methods (Goldberg and Zhu, 2006; Ando and Zhang, 2005a,b; Blei et al., 2003; Nigam et al., 2000; Joachims, 1999b) help to infer information from unlabeled data, because is more available than labeled data. (Gabrilovich and  Markovitch, 2007) introduce a method for enhancing machine learning algorithms with a large volume of extracted knowledge, mainly by existing induction techniques while enriching the language of representation, namely, exploring new feature spaces. In (Ruiz, Ariza, Ureñaa and Blázquezb, 2010) resent a classification of all conflation processes of geospatial databases from heterogeneous sources.

Base on above research, there might be possible to come up a useful tool for create a link to external source. The external source would base on the databases which is content lots of articles  generated by human. These databases include journal articles, websites, and also news. These Link will provide more information and understand for each word and sentence. We hope that it will be a good using experience reading articles with these links.
%%%%%%%%%%%%%%%%%%%%%%%%%%%%%%%%%%%%%%%%%%%%%%%%%%%%%%%%%%%%%%%%%%%%%%%%%%%%%%%%%%%
% The bibliography
%%%%%%%%%%%%%%%%%%%%%%%%%%%%%%%%%%%%%%%%%%%%%%%%%%%%%%%%%%%%%%%%%%%%%%%%%%%%%%%%%%%
%\bibliography{Bibliography_template} %Read the bibliography from a separate file

\end{document}      % End of the document

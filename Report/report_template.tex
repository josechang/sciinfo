\documentclass[a4paper]{article} % Document type

\ifx\pdfoutput\undefined
    %Use old Latex if PDFLatex does not work
   \usepackage[dvips]{graphicx}% To get graphics working
   \DeclareGraphicsExtensions{.eps} % Encapsulated PostScript
 \else
    %Use PDFLatex
   \usepackage[pdftex]{graphicx}% To get graphics working
   \DeclareGraphicsExtensions{.pdf,.jpg,.png,.mps} % Portable Document Format, Joint Photographic Experts Group, Portable Network Graphics, MetaPost
   \pdfcompresslevel=9
\fi

\usepackage{amsmath,amssymb}   % Contains mathematical symbols
\usepackage[ansinew]{inputenc} % Input encoding, identical to Windows 1252
\usepackage[english]{babel}    % Language
\usepackage[round,authoryear]{natbib}  %Nice author (year) citations
%\usepackage[square,numbers]{natbib}     %Nice numbered citations
%\bibliographystyle{unsrtnat}           %Unsorted bibliography
\bibliographystyle{plainnat}            %Sorted bibliography

\addtolength{\topmargin}{-20mm}% Removes 30mm from the top margin
\addtolength{\textheight}{10mm}% Adds it to the text height


\begin{document}               % Begins the document

\title{Report template for homework}
\author{First name Last name \\ student number \\ email} 
%\date{2010-10-10}             % If you want to set the date yourself.

\maketitle                     % Generates the title




%%%%%%%%%%%%%%%%%%%%%%%%%%%%%%%%%%%%%%%%%%%%%%%%%%%%%%%%%%%%%%%%%%%%%%%%%%%%%%%%%%%
% Instructions regarding the report
%%%%%%%%%%%%%%%%%%%%%%%%%%%%%%%%%%%%%%%%%%%%%%%%%%%%%%%%%%%%%%%%%%%%%%%%%%%%%%%%%%%

\section*{Problem}
\label{sec:prob}

Describe the problem in your own words and break it down into subproblems, unless the problem was alreadyan example, see Problem 1.

\section*{Solution}
\label{sec:sol}

Solve the subproblems in succession and give the solution to the overall problem at the end, followed by a senttables if they fosters the understanding of the solution. Each figure and table should include a caption, and all the figures and tables should be associated with a reference in the text and placed at the end of your report -- for example, see Figure~\ref{fig:prestanda} and Table~\ref{tab:JacobianMatrix}.  Each equation should only be written once and referred to by reference later when needed -- for example, see \eqref{eqn:PPSystem}. Equations that you do not refer to later may be left unlabeled. The report should be written using \LaTeX\ -- see \cite{Oetiker:2008:TheNotSoShortIntroductiontoLaTeXe} for an introduction.




%%%%%%%%%%%%%%%%%%%%%%%%%%%%%%%%%%%%%%%%%%%%%%%%%%%%%%%%%%%%%%%%%%%%%%%%%%%%%%%%%%%
% Example 1
%%%%%%%%%%%%%%%%%%%%%%%%%%%%%%%%%%%%%%%%%%%%%%%%%%%%%%%%%%%%%%%%%%%%%%%%%%%%%%%%%%%

\section*{Problem 1 -- Behavior of a nonlinear system}
\label{sec:prob1}
Study the behavior of the system
\begin{subequations}\label{eqn:PPSystem}
\begin{align}
    \frac{d x}{dt} & = y \label{eqn:PPSystem1} \\
    \frac{d y}{dt} & = -2x -2y -4 x^2 \label{eqn:PPSystem2}
\end{align}
\end{subequations}
with regard to its initial value and assess if the system is globally asymptotically stable.

We have chosen a solution strategy based on the phase portrait and broken down the problem into the following three subproblems:
\begin{enumerate}
  \item Draw the phase portrait of the system.
  \item How many equilibrium points does the system have, where are they located, and what type of equilibrium points are they?
  \item Is the system globally asymptotically stable?  That is, do all trajectories lead to the same equilibrium point?
\end{enumerate}

\section*{Solution 1 -- Behavior of a nonlinear system}
\label{sec:sol1}

\begin{enumerate}
  \item The phase portrait of the system described by \eqref{eqn:PPSystem} is shown in Figure~\ref{fig:pplane}.

  \item The system has two equilibrium points, \mbox{(0,0)} and \mbox{($-\frac{1}{2}$,0)}.  They were computed by setting the right hand side of \eqref{eqn:PPSystem} equal to zero
\begin{align*}
    \begin{array}{rl}
    y & =0  \\
    -2x -2y -4 x^2 & = 0
    \end{array} \; \Rightarrow
    \begin{array}{rl}
      y & =0 \\
      x (1+2x) & =0
    \end{array}.
\end{align*}
Both points are marked by large dots in Figure~\ref{fig:pplane}.

Let us now characterize the equilibrium points. The Hartman-Grobman theorem~\cite[p. 288]{Sastry:1999:Nonlinear-systems:-analysis-stability-and-control:xr} states that the behavior of a nonlinear system near a hyperbolic equilibrium is qualitatively the same as its linearization at the equilibrium point. An equilibrium is by definition hyperbolic if the real part of the eigenvalues of the Jacobian matrix are all non-zero. We will therefore linearize the system \eqref{eqn:PPSystem} around both equilibrium points, calculate the eigenvalues, check if the points are hyperbolic, and determine the type of the equilibria based on the eigenvalues.

The linearization of an autonomous nonlinear system $\dot{\textbf{x}} = \textbf{f}(\textbf{x})$ around an equilibrium point, given by $\textbf{f}(\textbf{x}_0) = 0$, is $\dot{\tilde{\textbf{x}}} = A \tilde{\textbf{x}}$, with $\tilde{\textbf{x}} = \textbf{x} - \textbf{x}_0$ and the Jacobian matrix $A = \left. \frac{\partial\textbf{f}}{\partial \textbf{x}}(\textbf{x}) \right|_{\textbf{x}=\textbf{x}_0}$. The Jacobian matrix is calculated to be
\begin{equation*}
    \frac{\partial\textbf{f}}{\partial \textbf{x}}(\textbf{x}) =
    \left[\begin{array}{cc}
    0 & 1 \\
    -2 -8x & -2
    \end{array}\right],
\end{equation*}
which for the two equilibrium points turns out to be
\begin{equation*}
    A_{(0,0)} =
    \left[\begin{array}{cc}
    0 & 1 \\
    -2 & -2
    \end{array}\right], \; A_{(-\frac{1}{2},0)} =
    \left[\begin{array}{cc}
    0 & 1 \\
    2 & -2
    \end{array}\right].
\end{equation*}
The eigenvalues of the Jacobian matrix are given by the characteristic equation
\begin{equation*}
    \det(\lambda I - A) = 0.
\end{equation*}
The characteristic equation of the system linearized around \mbox{(0,0)} is
\begin{equation*}
    \lambda^2 +2 \lambda + 2 = 0,
\end{equation*}
which gives the eigenvalues $\lambda_{1,2} = -1 \pm j$. The characteristic equation of the system linearized around \mbox{($-\frac{1}{2}$,0)} is
\begin{equation*}
    \lambda^2 +2 \lambda - 2 = 0,
\end{equation*}
which gives the eigenvalues $\lambda_1 = \sqrt{3} -1 \approx 0.73$ and $\lambda_2 = -1 -\sqrt{3} \approx -2.73$. Both equilibrium points are hence hyperbolic and we can characterize them based on the eigenvalues of the Jacobian matrix of the linearized systems.  The eigenvalues tell us that we have a stable focus at \mbox{(0,0)} and a saddle point at \mbox{($-\frac{1}{2}$,0)}. This is easily verified by looking at Figure~\ref{fig:pplane}.

  \item Only trajectories starting within the shaded region and its extension above the part of the phase portrait shown in Figure~\ref{fig:pplane} leads to the stable focus. It is therefore obvious that the system is not globally asymptotically stable.
\end{enumerate}

\noindent \textbf{Answer:} The system has two equilibrium points; a stable focus at \mbox{(0,0)} and a saddle point at \mbox{($-\frac{1}{2}$,0)}. It is not globally asymptotically stable.



%%%%%%%%%%%%%%%%%%%%%%%%%%%%%%%%%%%%%%%%%%%%%%%%%%%%%%%%%%%%%%%%%%%%%%%%%%%%%%%%%%%
% The bibliography
%%%%%%%%%%%%%%%%%%%%%%%%%%%%%%%%%%%%%%%%%%%%%%%%%%%%%%%%%%%%%%%%%%%%%%%%%%%%%%%%%%%
%\bibliography{Bibliography_template} %Read the bibliography from a separate file

\begin{thebibliography}{99}
\bibitem[Khalil(2002)]{Khalil:2002:Nonlinear-systems:vh}
Hassan~K Khalil.
\newblock \emph{Nonlinear systems}.
\newblock Prentice Hall, Upper Saddle river, 3. edition, 2002.
\newblock ISBN 0-13-067389-7.

\bibitem[Oetiker et~al.(2008)Oetiker, Partl, Hyna, and
  Schlegl]{Oetiker:2008:TheNotSoShortIntroductiontoLaTeXe}
Tobias Oetiker, Hubert Partl, Irene Hyna, and Elisabeth Schlegl.
\newblock \emph{The Not So Short Introduction to \LaTeXe}.
\newblock Oetiker, OETIKER+PARTNER AG, Aarweg 15, 4600 Olten, Switzerland,
  2008.
\newblock http://www.ctan.org/info/lshort/.

\bibitem[Sastry(1999)]{Sastry:1999:Nonlinear-systems:-analysis-stability-and-c%
ontrol:xr}
Shankar Sastry.
\newblock \emph{Nonlinear systems: analysis, stability, and control},
  volume~10.
\newblock Springer, New York, N.Y., 1999.
\newblock ISBN 0-387-98513-1.
\end{thebibliography}



%%%%%%%%%%%%%%%%%%%%%%%%%%%%%%%%%%%%%%%%%%%%%%%%%%%%%%%%%%%%%%%%%%%%%%%%%%%%%%%%%%%
% Place your figures and tables at the end of the document starting on a new page
%%%%%%%%%%%%%%%%%%%%%%%%%%%%%%%%%%%%%%%%%%%%%%%%%%%%%%%%%%%%%%%%%%%%%%%%%%%%%%%%%%%
\clearpage % Ends the current page and causes all figures and tables to be printed

\begin{figure*}[p] % The * makes the figure span both columns, p places the figure on a float page
  \begin{center}
    \includegraphics[scale=1.0]{Stepresponse}
  \end{center}
  \caption{Step response of the two systems. The figure should have a caption that briefly explains what it shows. The color, symbol and line type should be selected such that the lines can be distinguished when printed in black and white. The lines should be annotated either in the plot, in a legend, or in the caption. The axis should have a label in which the units are given. The figure should be sufficiently large, clear, easy to understand and only contain essential information.}
  \label{fig:prestanda}
\end{figure*}

\begin{table*}[p] % The * makes the table span both columns, p places the table on a float page
\caption{Jacobian matrix $A$ of the linear system $\dot{\textbf{x}} = A \textbf{x}$. The table should have a caption that briefly explains what it shows.}
\label{tab:JacobianMatrix}
\begin{center}
\begin{tabular}{@{\vrule height 10.5pt depth4pt  width0pt}|c|c|c|c|}
    \hline
     $-2.46$ & $0$ & $-1.73$ & $0$ \\ \hline
     $0$ & $-2.553$ & $0$ & $2.774$ \\ \hline
     $0$ & $6.172$ & $-10$ & $7.333$ \\ \hline
     $1.767$ & $-0.357$ & $5.714$ & $-6.074$ \\ \hline
\end{tabular}
\end{center}
\end{table*}


\begin{figure*}[p] % The * makes the figure span both columns, p places the figure on a float page
  \begin{center}
    \includegraphics[width = 1.0\textwidth]{Phase_portrait_with_regionmark}
  \end{center}
  \caption{Phase portrait of the system in \eqref{eqn:PPSystem}, which has a stable focus at \mbox{(0,0)} and a saddle point at \mbox{(-0.5,0)}. Both equilibriums are marked by large dots and selected trajectories are marked by solid lines. Trajectories starting within the shaded region end at the stable focus. This figure was generated in
  Matlab (\texttt{http://www.mathworks.com}) using pplane7 (\texttt{http://math.rice.edu/\textasciitilde dfield/}).}
  \label{fig:pplane}
\end{figure*}

\end{document}      % End of the document

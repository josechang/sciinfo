\documentclass[a4paper]{article} % Document type

\ifx\pdfoutput\undefined
    %Use old Latex if PDFLatex does not work
   \usepackage[dvips]{graphicx}% To get graphics working
   \DeclareGraphicsExtensions{.eps} % Encapsulated PostScript
 \else
    %Use PDFLatex
   \usepackage[pdftex]{graphicx}% To get graphics working
   \DeclareGraphicsExtensions{.pdf,.jpg,.png,.mps} % Portable Document Format, Joint Photographic Experts Group, Portable Network Graphics, MetaPost
   \pdfcompresslevel=9
\fi

\usepackage[bookmarks=true,bookmarksnumbered=true,hypertexnames=true,breaklinks=true,colorlinks=true]{hyperref}
\usepackage{amsmath,amssymb}   % Contains mathematical symbols
\usepackage[ansinew]{inputenc} % Input encoding, identical to Windows 1252
\usepackage[english]{babel}    % Language
\usepackage[round,authoryear]{natbib}  %Nice author (year) citations
%\usepackage[square,numbers]{natbib}     %Nice numbered citations
%\bibliographystyle{unsrtnat}           %Unsorted bibliography
\bibliographystyle{plainnat}            %Sorted bibliography

\addtolength{\topmargin}{-20mm}% Removes 30mm from the top margin
\addtolength{\textheight}{10mm}% Adds it to the text height


\begin{document}               % Begins the document

\title{Information Retrieval Report}
\author{First name Last name \\ student number \\ email} 
%\date{2010-10-10}             % If you want to set the date yourself.

\maketitle                     % Generates the title




%%%%%%%%%%%%%%%%%%%%%%%%%%%%%%%%%%%%%%%%%%%%%%%%%%%%%%%%%%%%%%%%%%%%%%%%%%%%%%%%%%%
% Abstract of the whole project
%%%%%%%%%%%%%%%%%%%%%%%%%%%%%%%%%%%%%%%%%%%%%%%%%%%%%%%%%%%%%%%%%%%%%%%%%%%%%%%%%%%

\section*{Abstract}
\label{abstract}

Whole project is seperaed to four major parts:
\begin{enumerate}
  \item Create a data base of open access full text articles responded by team Wolverine
  \item Create a complementary XML structure for metadata that contains all information in the PDF and can be show in Utopia responded by team Eagle unit
  \item Automatic creation of metadata/markup by use of natural language processing of full text articles responded by team Union
  \item Automatic creation of links from full text articles to external sources responded by team Hanky Phanky
\end{enumerate}

%%%%%%%%%%%%%%%%%%%%%%%%%%%%%%%%%%%%%%%%%%%%%%%%%%%%%%%%%%%%%%%%%%%%%%%%%%%%%%%%%%%
% Responsibility of Wolverine
%%%%%%%%%%%%%%%%%%%%%%%%%%%%%%%%%%%%%%%%%%%%%%%%%%%%%%%%%%%%%%%%%%%%%%%%%%%%%%%%%%%

\section*{Create a data base of open access full text article}
\label{task1}

Goal for this section is to build up a database which downloads articles from open access databases using web crawler or web robot. And setting up a system that can allow the users to access the data in our database. I'd like to seperate this section into six features, then I'll give several suggestions based on each feature. 

\subsection*{Feature 1 --Administrator}
\label{task1:part1}

In this part, the main consideration is the relationship between user and administrator. As an administrator's point, I'd like to give the best product to the consumer. On the other hand, users want to have convenient searching engine to get in touch with the knowledge. For this purpose, I'm pressure to show two suggestions for enhancing the relationship between users and producers. First, set up a popular-word-ranking system. When I search in something I don't know its name but know it in some specific area, such as stem cell in medical area. In this moment, the popular-word-ranking system will help me searching by giving me some keywords. Then, I would be easily to use this database. Of course, the popular-word-ranking system is based on users' searches and experts'suggestions to change the key words. So, the system can be trusted. 

Second, build up an area in the database and let user to change the area to what they want it to be. The idea is referenced from well-known media, Wikimedia. It's so called "personalized searching". By doing this, it would help user idealize the system to what they want and suggest administrator the service which clients really want. It would lead to a win-win situation.

\subsection*{Feature 2 -- Web Crawler}
\label{task1:feature2}

Study the behavior of the system with regard to its initial value and assess if the system is globally asymptotically stable.
 
\subsection*{Feature 3 -- User account}
\label{task1:feature3}

In this part, the main issue is how to create a user account that can connect between user and database. But the more important thing is to make sure database will not collapse by user who is not allowed to access to core part of database.

To protect the database system security and privilege, this study introduces two methods for user account, principle of least privilege and role-based access control respectively. The principle of least privilege, also known as the principle of minimal privilege, means giving a user account only those those privileges which are essential to that user`s work \url{https://en.wikipedia.org/wiki/Principle_of_least_privilege} (2016/03/16). The role-based access control is a policy neutral access control mechanism defined around privileges and roles. It can implement discretionary access control (DAC) or mandatory access control (MAC). The role-based access control is very easy to do user assignments as the components of this policy, such as role-permissions, etc. That is why it sometimes referred to as role-based security \url{https://en.wikipedia.org/wiki/Role-based_access_control} (2016/03/16). The information and resources would not in danger due to these two methods will filter user depend on their authority and only allow the legitimate user to access.

\subsection*{Feature 4 -- User Interface}
\label{task1:feature4}

Study the behavior of the system with regard to its initial value and assess if the system is globally asymptotically stable.

\subsection*{Feature 5 -- Data Storage and Search Methods}
\label{task1:feature5}

The organization of data inside a database management system(DBMS) and retrieval methods is based on the database storage structure such as tables and indexes. 
There are several types of database storage structure such as XML, a textual data format. 
This advantage is self-describing and flexible in organizing data.\cite{one}Several considerations of data storage include right space allocation techniques, data compression techniques (if necessary), security and encryption and the access path to retrieve the data. 
Therefore, DBMS software will provide some method to optimize and  minimum storage space of a database.

\subsection*{Feature 6 -- Database Management Systems}
\label{task1:feature6}

The relational database model was proposed by Edgar Codd in 1970, but because of the technological requirements it was not universal at that time. It was until 1980s that the first commercial relational database management systems began to appear. 
A database management system (DBMS) is a computer software application that interacts with the user, other applications, and the database itself to capture and analyze data. Well-known DBMSs include MySQL, PostgreSQL, Microsoft SQL Server, Oracle, Sybase and IBM DB2. 
Queries are the main way in which users retrieve information from a database. Most database management systems use a standard system called Structured Query Language (SQL) to query their tables. SQL is a structured form of English and resembles many programming languages, although most systems provide a graphical user-interface to generate the SQL code. And the most popular database systems since the 1980s have all supported the relational model as represented by the SQL language.



%%%%%%%%%%%%%%%%%%%%%%%%%%%%%%%%%%%%%%%%%%%%%%%%%%%%%%%%%%%%%%%%%%%%%%%%%%%%%%%%%%%
% Responsibility of Eagle unit
%%%%%%%%%%%%%%%%%%%%%%%%%%%%%%%%%%%%%%%%%%%%%%%%%%%%%%%%%%%%%%%%%%%%%%%%%%%%%%%%%%%

\section*{Task 1 -- Building up the database}
\label{task1}

Goal for this section is to build up a database which downloads articles from open access databases using web crawler or web robot. And setting up a system that can allow the users to access the data in our database.

\subsection*{Feature 1 --Administrator}
\label{task1:feature1}

Study the behavior of the system with regard to its initial value and assess if the system is globally asymptotically stable.

\subsection*{Feature 2 -- Web Crawler}
\label{task1:feature2}

Study the behavior of the system with regard to its initial value and assess if the system is globally asymptotically stable.

\subsection*{Feature 3 -- User account}
\label{task1:feature3}

Study the behavior of the system with regard to its initial value and assess if the system is globally asymptotically stable.

\subsection*{Feature 4 -- User Interface}
\label{task1:feature4}

Study the behavior of the system with regard to its initial value and assess if the system is globally asymptotically stable.

\subsection*{Feature 5 -- Data Storage and Search Methods}
\label{task1:feature5}

Study the behavior of the system with regard to its initial value and assess if the system is globally asymptotically stable.

\subsection*{Feature 6 -- Database}
\label{task1:feature6}

Study the behavior of the system with regard to its initial value and assess if the system is globally asymptotically stable.


%%%%%%%%%%%%%%%%%%%%%%%%%%%%%%%%%%%%%%%%%%%%%%%%%%%%%%%%%%%%%%%%%%%%%%%%%%%%%%%%%%%
% Responsibility of Union
%%%%%%%%%%%%%%%%%%%%%%%%%%%%%%%%%%%%%%%%%%%%%%%%%%%%%%%%%%%%%%%%%%%%%%%%%%%%%%%%%%%

\section*{Task 1 -- Building up the database}
\label{task1}

Goal for this section is to build up a database which downloads articles from open access databases using web crawler or web robot. And setting up a system that can allow the users to access the data in our database.

\subsection*{Feature 1 --Administrator}
\label{task1:feature1}

Study the behavior of the system with regard to its initial value and assess if the system is globally asymptotically stable.

\subsection*{Feature 2 -- Web Crawler}
\label{task1:feature2}

Study the behavior of the system with regard to its initial value and assess if the system is globally asymptotically stable.

\subsection*{Feature 3 -- User account}
\label{task1:feature3}

Study the behavior of the system with regard to its initial value and assess if the system is globally asymptotically stable.

\subsection*{Feature 4 -- User Interface}
\label{task1:feature4}

Study the behavior of the system with regard to its initial value and assess if the system is globally asymptotically stable.

\subsection*{Feature 5 -- Data Storage and Search Methods}
\label{task1:feature5}

Study the behavior of the system with regard to its initial value and assess if the system is globally asymptotically stable.

\subsection*{Feature 6 -- Database}
\label{task1:feature6}

Study the behavior of the system with regard to its initial value and assess if the system is globally asymptotically stable.


%%%%%%%%%%%%%%%%%%%%%%%%%%%%%%%%%%%%%%%%%%%%%%%%%%%%%%%%%%%%%%%%%%%%%%%%%%%%%%%%%%%
% Responsibility of Hanky Phanky
%%%%%%%%%%%%%%%%%%%%%%%%%%%%%%%%%%%%%%%%%%%%%%%%%%%%%%%%%%%%%%%%%%%%%%%%%%%%%%%%%%%

\section*{Task 1 -- Building up the database}
\label{task1}

Goal for this section is to build up a database which downloads articles from open access databases using web crawler or web robot. And setting up a system that can allow the users to access the data in our database.

\subsection*{Feature 1 --Administrator}
\label{task1:feature1}

Study the behavior of the system with regard to its initial value and assess if the system is globally asymptotically stable.

\subsection*{Feature 2 -- Web Crawler}
\label{task1:feature2}

Study the behavior of the system with regard to its initial value and assess if the system is globally asymptotically stable.

\subsection*{Feature 3 -- User account}
\label{task1:feature3}

Study the behavior of the system with regard to its initial value and assess if the system is globally asymptotically stable.

\subsection*{Feature 4 -- User Interface}
\label{task1:feature4}

Study the behavior of the system with regard to its initial value and assess if the system is globally asymptotically stable.

\subsection*{Feature 5 -- Data Storage and Search Methods}
\label{task1:feature5}

Study the behavior of the system with regard to its initial value and assess if the system is globally asymptotically stable.

\subsection*{Feature 6 -- Database}
\label{task1:feature6}

Study the behavior of the system with regard to its initial value and assess if the system is globally asymptotically stable.


%%%%%%%%%%%%%%%%%%%%%%%%%%%%%%%%%%%%%%%%%%%%%%%%%%%%%%%%%%%%%%%%%%%%%%%%%%%%%%%%%%%
% The bibliography
%%%%%%%%%%%%%%%%%%%%%%%%%%%%%%%%%%%%%%%%%%%%%%%%%%%%%%%%%%%%%%%%%%%%%%%%%%%%%%%%%%%
%\bibliography{Bibliography_template} %Read the bibliography from a separate file

%% I would strongly recommend you to use Mendeley desktop or a similar reference manager 
%% that can generate a bibtex file so that you don't need to insert references manually. 
%% It will save you a lot of time. Bests, Prof. Nordling

\begin{thebibliography}{99}
\bibitem[Khalil(2002)]{Khalil:2002:Nonlinear-systems:vh}
Hassan~K Khalil.
\newblock \emph{Nonlinear systems}.
\newblock Prentice Hall, Upper Saddle river, 3. edition, 2002.
\newblock ISBN 0-13-067389-7.

\bibitem[Oetiker et~al.(2008)Oetiker, Partl, Hyna, and
  Schlegl]{Oetiker:2008:TheNotSoShortIntroductiontoLaTeXe}
Tobias Oetiker, Hubert Partl, Irene Hyna, and Elisabeth Schlegl.
\newblock \emph{The Not So Short Introduction to \LaTeXe}.
\newblock Oetiker, OETIKER+PARTNER AG, Aarweg 15, 4600 Olten, Switzerland,
  2008.
\newblock http://www.ctan.org/info/lshort/.

\bibitem[Sastry(1999)]{Sastry:1999:Nonlinear-systems:-analysis-stability-and-c%
ontrol:xr}
Shankar Sastry.
\newblock \emph{Nonlinear systems: analysis, stability, and control},
  volume~10.
\newblock Springer, New York, N.Y., 1999.
\newblock ISBN 0-387-98513-1.
\end{thebibliography}



%%%%%%%%%%%%%%%%%%%%%%%%%%%%%%%%%%%%%%%%%%%%%%%%%%%%%%%%%%%%%%%%%%%%%%%%%%%%%%%%%%%
% Place your figures and tables at the end of the document starting on a new page
%%%%%%%%%%%%%%%%%%%%%%%%%%%%%%%%%%%%%%%%%%%%%%%%%%%%%%%%%%%%%%%%%%%%%%%%%%%%%%%%%%%
\clearpage % Ends the current page and causes all figures and tables to be printed

\end{document}      % End of the document

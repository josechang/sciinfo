\documentclass[a4paper]{article}
%%%%%%%%%%%%%%%%%%%%%%%%%%%%%%%%%%%%%%%%%%%%%%%%%%%%%%%%%%%%%%%%%%%
% Packages using
%%%%%%%%%%%%%%%%%%%%%%%%%%%%%%%%%%%%%%%%%%%%%%%%%%%%%%%%%%%%%%%%%%%
\ifx\pdfoutput\undefined
	\usepackage[dvips]{graphicx}
	\DeclareGraphicsExtensions{.eps}
\else
	\usepackage[pdftex]{graphicx}
	\DeclareGraphicsExtensions{.pdf,.jpg,.png,.mps}
\pdfcompresslevel=9
\fi
\usepackage[bookmarks=true,bookmarksnumbered=true,hypertexnames=true,breaklinks=true,colorlinks=true]{hyperref}
\usepackage{amsmath,amssymb}
\usepackage[ansinew]{inputenc}
\usepackage[english]{babel}
\usepackage{indentfirst}
\addtolength{\topmargin}{-20mm}
\addtolength{\textheight}{10mm}

\begin{document}               % Begins the document

\title{Structure of XML Metadata}
\author{EagleUnit \\ (I-Chieh, Chinweze, Henry, Ray, Jones, Piyarul)} 
%\date{2010-10-10}             % If you want to set the date yourself.

\maketitle                     % Generates the title




%%%%%%%%%%%%%%%%%%%%%%%%%%%%%%%%%%%%%%%%%%%%%%%%%%%%%%%%%%%%%%%%%%%%%%%%%%%%%%%%%%%
% Team:
% EagleUnit
% Members: 
%  I-Chei, Chinweze, Jones, Ray, Hunry, Piyarul, Tim
% Relative files:
% 
% Note:    
% 
%%%%%%%%%%%%%%%%%%%%%%%%%%%%%%%%%%%%%%%%%%%%%%%%%%%%%%%%%%%%%%%%%%%%%%%%%%%%%%%%%%%

\section*{XML metadata structure}
\subsection{ABSTRACT}
\label{sec:abs}
Our responsipility is to construct an XML metadata structure.

\subsection{BACKGROUND}
%%%%%%%%%%%%%%%%%%%%%%%%%%%%%%%%%%%%%%%%%%%%%%%%%%%%%%%%%%%%%%%%%%%%%%%%%%%%%%%%%%%
% 1. Different standards of metadata
%%%%%%%%%%%%%%%%%%%%%%%%%%%%%%%%%%%%%%%%%%%%%%%%%%%%%%%%%%%%%%%%%%%%%%%%%%%%%%%%%%%

\subsubsection{Different standards of metadata}
\label{sec:mets}
There's many different standards existing to describe the metadata in different fields and applications. We list 5 of them and give a brief introduction to each one.

\begin{enumerate}
	\item METS\\
	{\bf Introduction}\\
	Metadata Encoding and Transmission Standard (METS) similar to XML encoding format for storing the descriptive, administrative, structural and behavioral metadata needed to manage complex digital objects in open and standardized way.
	
	In 1990s, Making of America II (MOA2) project was proposed to share vision between national digital libraries which provides a means for the Digital Library Federation (DLF) to investigate, refine, and recommend metadata elements and encodings used to discover, display, and navigate digital archival objects. MOA2 DTD was created to test MOA2 project.
	
	However, MOA2 DTD was limited in several ways. It provided no flexibility in terms of the exact metadata elements to be used for descriptive, administrative and structural metadata. Also, limited in scope to support for text and still image materials and no attempt to support time-based media such as audio or video materials. To solve those problems led to the creation of METS.
	
	{\bf Adventage}
	\begin{enumerate}
		\item testA
		\item testB
	\end{enumerate}	
	{\bf Disadventage}
	\begin{enumerate}
		\item testA
		\item testB
	\end{enumerate}
	{\bf Conclusion}\\	
	
	%%%%%%%%%%%%%%%%%%%%%%%%%%%%%%%%%%%%%%%%%%%%%%%%%%%%%%%%%%%%%%%%%%%%%%%%%%%%%%%%%%%
	\item MODS\\
	{\bf Introduction}\\
	
	{\bf Adventage}
	\begin{enumerate}
		\item testA
		\item testB
	\end{enumerate}	
	{\bf Disadventage}
	\begin{enumerate}
		\item testA
		\item testB
	\end{enumerate}
	{\bf Conclusion}\\
	
	%%%%%%%%%%%%%%%%%%%%%%%%%%%%%%%%%%%%%%%%%%%%%%%%%%%%%%%%%%%%%%%%%%%%%%%%%%%%%%%%%%%
	\item METS+MODS+PREMIS\\
	{\bf Introduction}\\
	
	{\bf Adventage}
	\begin{enumerate}
		\item testA
		\item testB
	\end{enumerate}	
	{\bf Disadventage}
	\begin{enumerate}
		\item testA
		\item testB
	\end{enumerate}
	{\bf Conclusion}\\
	
	%%%%%%%%%%%%%%%%%%%%%%%%%%%%%%%%%%%%%%%%%%%%%%%%%%%%%%%%%%%%%%%%%%%%%%%%%%%%%%%%%%%
	\item MARC 21\\
	{\bf Introduction}\\
	
	{\bf Adventage}
	\begin{enumerate}
		\item testA
		\item testB	
	\end{enumerate}	
	{\bf Disadventage}
	\begin{enumerate}
		\item testA
		\item testB
	\end{enumerate}
	{\bf Conclusion}\\
	
	%%%%%%%%%%%%%%%%%%%%%%%%%%%%%%%%%%%%%%%%%%%%%%%%%%%%%%%%%%%%%%%%%%%%%%%%%%%%%%%%%%%	
	\item MARCXML\\
	{\bf Introduction}\\
	
	{\bf Adventage}
	\begin{enumerate}
		\item testA
		\item testB
	\end{enumerate}	
	{\bf Disadventage}
	\begin{enumerate}
		\item testA
		\item testB
	\end{enumerate}
	{\bf Conclusion}\\
	
	%%%%%%%%%%%%%%%%%%%%%%%%%%%%%%%%%%%%%%%%%%%%%%%%%%%%%%%%%%%%%%%%%%%%%%%%%%%%%%%%%%%
	\item Dublin Core\\
	{\bf Introduction}\\
	
	{\bf Adventage}
	\begin{enumerate}
		\item testA
		\item testB
	\end{enumerate}	
	{\bf Disadventage}
	\begin{enumerate}
		\item testA
		\item testB
	\end{enumerate}
	{\bf Conclusion}\\
	
	%%%%%%%%%%%%%%%%%%%%%%%%%%%%%%%%%%%%%%%%%%%%%%%%%%%%%%%%%%%%%%%%%%%%%%%%%%%%%%%%%%%
	\item IAFA/Whois++ Templates\\
	{\bf Introduction}\\
	
	{\bf Adventage}
	\begin{enumerate}
		\item testA
		\item testB
	\end{enumerate}	
	{\bf Disadventage}
	\begin{enumerate}
		\item testA
		\item testB
	\end{enumerate}
	{\bf Conclusion}\\	
	
\end{enumerate}

More detailed introduction could be found in {\bf Metadata for learning materials (Barker 2010)} \cite{barker2010metadata} and {\bf Review of metadata formats (Heery 996)} \cite{heery1996review}.

%%%%%%%%%%%%%%%%%%%%%%%%%%%%%%%%%%%%%%%%%%%%%%%%%%%%%%%%%%%%%%%%%%%%%%%%%%%%%%%%%%%
% 2. Necessary elements of XML metadata
%%%%%%%%%%%%%%%%%%%%%%%%%%%%%%%%%%%%%%%%%%%%%%%%%%%%%%%%%%%%%%%%%%%%%%%%%%%%%%%%%%%

\subsubsection*{2. Necessary elements of XML metadata with DTD}
\label{sec:mets}
{\bf Ruey-Shun}\cite{chen2003developing} suggest that an XML metadata discribed according the DTD include three necessary elements:
\begin{enumerate}
	\item Structure\\
	The major execution ability of structure includes parser for well-formed XML and
	valid DTD structure, authoring tool for editing.
	
	\item Depth\\
	Basically, there are two sorts of fields: Fixed-length fields and variable fields.
	Fixed-length fields are general types and character-indication types Sub-field, whether
	fixed-length fields or variable fields, might contain both fixed-length fields and
	variable fields. According to the reason above, the process ability of the system has to
	cover the situation
	
	\item Scope\\
	The connections must involve simple object, time, space, people, and event. 
\end{enumerate}


%%%%%%%%%%%%%%%%%%%%%%%%%%%%%%%%%%%%%%%%%%%%%%%%%%%%%%%%%%%%%%%%%%%%%%%%%%%%%%%%%
% The bibliography
%%%%%%%%%%%%%%%%%%%%%%%%%%%%%%%%%%%%%%%%%%%%%%%%%%%%%%%%%%%%%%%%%%%%%%%%%%%%%%%%%

	\bibliography{library}
	\clearpage 


%%%%%%%%%%%%%%%%%%%%%%%%%%%%%%%%%%%%%%%%%%%%%%%%%%%%%%%%%%%%%%%%%%%%%%%%%%%%%%%%%
% figures
%%%%%%%%%%%%%%%%%%%%%%%%%%%%%%%%%%%%%%%%%%%%%%%%%%%%%%%%%%%%%%%%%%%%%%%%%%%%%%%%%
\clearpage % Ends the current page and causes all figures and tables to be printed



\end{document}      % End of the document

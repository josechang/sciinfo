\documentclass[a4paper]{article}
%%%%%%%%%%%%%%%%%%%%%%%%%%%%%%%%%%%%%%%%%%%%%%%%%%%%%%%%%%%%%%%%%%%
% Packages using
%%%%%%%%%%%%%%%%%%%%%%%%%%%%%%%%%%%%%%%%%%%%%%%%%%%%%%%%%%%%%%%%%%%
\ifx\pdfoutput\undefined
	\usepackage[dvips]{graphicx}
	\DeclareGraphicsExtensions{.eps}
\else
	\usepackage[pdftex]{graphicx}
	\DeclareGraphicsExtensions{.pdf,.jpg,.png,.mps}
\pdfcompresslevel=9
\fi
\usepackage[bookmarks=true,bookmarksnumbered=true,hypertexnames=true,breaklinks=true,colorlinks=true]{hyperref}
\usepackage{amsmath,amssymb}
\usepackage[ansinew]{inputenc}
\usepackage[english]{babel}
\usepackage{indentfirst}
\addtolength{\topmargin}{-20mm}
\addtolength{\textheight}{10mm}

\begin{document}               % Begins the document

\title{Structure of XML Metadata}
\author{EagleUnit \\ (I-Chieh, Chinweze, Henry, Ray, Jones, Piyarul)} 
%\date{2010-10-10}             % If you want to set the date yourself.

\maketitle                     % Generates the title




%%%%%%%%%%%%%%%%%%%%%%%%%%%%%%%%%%%%%%%%%%%%%%%%%%%%%%%%%%%%%%%%%%%%%%%%%%%%%%%%%%%
% Team:
% EagleUnit
% Members: 
%  I-Chei, Chinweze, Jones, Ray, Hunry, Piyarul, Tim
% Relative files:
% 
% Note:    
% 
%%%%%%%%%%%%%%%%%%%%%%%%%%%%%%%%%%%%%%%%%%%%%%%%%%%%%%%%%%%%%%%%%%%%%%%%%%%%%%%%%%%

\section*{XML metadata structure}
\subsection{ABSTRACT}
\label{sec:abs}
Our responsibility is to build a metadata schema which contains all information in the PDF and program it in a XML structure. This article present an overview of standards for metadata related to our responsibility. We review a number of metadata schemas in use today, including MODS, METS, METS+MODS+PREMIS, MARC21, MARC XML, and show the overviews and the pros and cons of each schema. Finally, through the examination of characteristics of these metadata schemas, we try to rank them and to suggest the optimum standard for our work.
\subsection{INTRODUCTION}
\subsection{BACKGROUND}
%%%%%%%%%%%%%%%%%%%%%%%%%%%%%%%%%%%%%%%%%%%%%%%%%%%%%%%%%%%%%%%%%%%%%%%%%%%%%%%%%%%
% 1. Different standards of metadata
%%%%%%%%%%%%%%%%%%%%%%%%%%%%%%%%%%%%%%%%%%%%%%%%%%%%%%%%%%%%%%%%%%%%%%%%%%%%%%%%%%%

\subsubsection{Different standards of metadata}
\label{sec:mets}
There's many different standards existing to describe the metadata in different fields and applications. We list 5 of them and give a brief introduction to each one.

\begin{enumerate}
	\item METS\\
	{\bf Introduction}\\
	Metadata Encoding and Transmission Standard (METS) similar to XML encoding format for storing the descriptive, administrative, structural and behavioral metadata needed to manage complex digital objects in open and standardized way.
	
	In 1990s, Making of America II (MOA2) project was proposed to share vision between national digital libraries which provides a means for the Digital Library Federation (DLF) to investigate, refine, and recommend metadata elements and encodings used to discover, display, and navigate digital archival objects. MOA2 DTD was created to test MOA2 project.
	
	However, MOA2 DTD was limited in several ways. It provided no flexibility in terms of the exact metadata elements to be used for descriptive, administrative and structural metadata. Also, limited in scope to support for text and still image materials and no attempt to support time-based media such as audio or video materials. To solve those problems led to the creation of METS.
	
	{\bf Adventage}
	\begin{enumerate}
		\item testA
		\item testB
	\end{enumerate}	
	{\bf Disadventage}
	\begin{enumerate}
		\item testA
		\item testB
	\end{enumerate}
	{\bf Conclusion}\\	
	
	%%%%%%%%%%%%%%%%%%%%%%%%%%%%%%%%%%%%%%%%%%%%%%%%%%%%%%%%%%%%%%%%%%%%%%%%%%%%%%%%%%%
	\item MODS\\
	{\bf Introduction}\\
	Metadata Object and Description Schema (MODS) is a schema for a bibliographic element set that may be used for a variety of purposes, and particularly for library applications. It is a MARC-compatible XML schema for encoding descriptive data. \\
	The Library of Congress' Network Development and MARC Standards Office, with interested experts, developed the Metadata Object Description Schema (MODS) in 2002 for a bibliographic element set that may be used for a variety of purposes, and particularly for library applications. As an XML schema it is intended to be able to carry selected data from existing MARC 21 records as well as to enable the creation of original resource description records. It includes a subset of MARC fields and uses language-based tags rather than numeric ones, in some cases regrouping elements from the MARC 21 bibliographic format. As of May 2015 this schema is in its third version (version 3.6). MODS is expressed using the XML schema language of the World Wide Web Consortium. The standard is maintained by the MODS Editorial Committee with support from the Network Development and MARC Standards Office of the Library of Congress.\\
	MODS could potentially be used as follows :\\
	� as an SRU specified format\\
	� as an extension schema to METS\\
	� to represent metadata for harvesting\\
	� for original resource description in XML syntax\\
	� for representing a simplified MARC record in XML\\
	� for metadata in XML that may be packaged with an electronic resource\\
	Features of MODS :\\
	� The elements generally inherit the semantics of MARC\\
	� Some data has been repackaged; in some cases what is in several data elements in MARC may be brought together into one in MODS\\
	� MODS does not assume the use of any specific cataloging code\\
	� Several elements have an optional ID attribute to facilitate linking at the element level.\\
	{\bf Adventage}
	\begin{enumerate}
		\item The element set is richer than Dublin Core.
		\item The element set is more compatible with library data than ONIX.
		\item The schema is more end user oriented than the full MARCXML schema.
		\item The element set is simpler than the full MARC format.
	\end{enumerate}	
	{\bf Disadventage}
	\begin{enumerate}
		\item An original MARC 21 record converted to MODS may not convert back to MARC 21 in its entirety without some loss of specificity in tagging or loss of data.
		\item In some cases if reconverted into MARC 21, the data may not be placed in exactly the same field that it started in because a MARC field may have been mapped to a more general one in MODS.
		\item MODS does not include business rules for populating the elements.
		\item Additional instructions would need to be provided for conversion details.
	\end{enumerate}
	{\bf Conclusion}\\
	MODS has a high level of compatibility with MARC records because it inherits the semantics of the equivalent data elements in the MARC 21 bibliographic format. It may be used for original resource description that allows for rich description that is generally compatible with existing library data and is expressed in XML syntax. Because it includes a subset of MARC fields and repackages some of them, it is particularly useful for technician input.\\
	An additional use of MODS is as an extension schema for descriptive matadata for METS object.
	%%%%%%%%%%%%%%%%%%%%%%%%%%%%%%%%%%%%%%%%%%%%%%%%%%%%%%%%%%%%%%%%%%%%%%%%%%%%%%%%%%%
	\item METS+MODS+PREMIS\\
	{\bf Introduction}\\
	
	{\bf Adventage}
	\begin{enumerate}
		\item testA
		\item testB
	\end{enumerate}	
	{\bf Disadventage}
	\begin{enumerate}
		\item testA
		\item testB
	\end{enumerate}
	{\bf Conclusion}\\
	
	%%%%%%%%%%%%%%%%%%%%%%%%%%%%%%%%%%%%%%%%%%%%%%%%%%%%%%%%%%%%%%%%%%%%%%%%%%%%%%%%%%%
	\item MARC 21\\
	{\bf Introduction}\\
	
	{\bf Adventage}
	\begin{enumerate}
		\item testA
		\item testB	
	\end{enumerate}	
	{\bf Disadventage}
	\begin{enumerate}
		\item testA
		\item testB
	\end{enumerate}
	{\bf Conclusion}\\
	
	%%%%%%%%%%%%%%%%%%%%%%%%%%%%%%%%%%%%%%%%%%%%%%%%%%%%%%%%%%%%%%%%%%%%%%%%%%%%%%%%%%%	
	\item MARCXML\\
	{\bf Introduction}\\
	To make up the less of internet compatibility of MARC 21, the Library of Congress developed an XML schema based on it. Which the schema is the MARCXML standard. The purpose of MARCXML is to build a metadata format with a simple, extensible and flexible structure, which can be presented in XML stylesheets. Since MARCXML was design to converged data from MARC 21, the structure and performance are very similar between these two standards.  \\
	{\bf Adventage}
	\begin{enumerate}
		\item Used in XML directly
		\item Easily work with MARC 21 system.
	\end{enumerate}	
	{\bf Disadventage}
	\begin{enumerate}
		\item The disadventage of MARC 21 can be almost totally found on MARCXML, except the ability of internet apllication.
	\end{enumerate}
	{\bf Conclusion}\\
	Since our responsibility is to construct an XML schema, MARCXML will be our choice if MARC 21 becomes the standard we work with.
	
	%%%%%%%%%%%%%%%%%%%%%%%%%%%%%%%%%%%%%%%%%%%%%%%%%%%%%%%%%%%%%%%%%%%%%%%%%%%%%%%%%%%
	\item Dublin Core\\
	{\bf Introduction}\\
	
	{\bf Adventage}
	\begin{enumerate}
		\item testA
		\item testB
	\end{enumerate}	
	{\bf Disadventage}
	\begin{enumerate}
		\item testA
		\item testB
	\end{enumerate}
	{\bf Conclusion}\\
	
	%%%%%%%%%%%%%%%%%%%%%%%%%%%%%%%%%%%%%%%%%%%%%%%%%%%%%%%%%%%%%%%%%%%%%%%%%%%%%%%%%%%
	\item IAFA/Whois++ Templates\\
	{\bf Introduction}\\
	
	{\bf Adventage}
	\begin{enumerate}
		\item testA
		\item testB
	\end{enumerate}	
	{\bf Disadventage}
	\begin{enumerate}
		\item testA
		\item testB
	\end{enumerate}
	{\bf Conclusion}\\	
	
\end{enumerate}

More detailed introduction could be found in {\bf Metadata for learning materials (Barker 2010)} \cite{barker2010metadata} and {\bf Review of metadata formats (Heery 996)} \cite{heery1996review}.

%%%%%%%%%%%%%%%%%%%%%%%%%%%%%%%%%%%%%%%%%%%%%%%%%%%%%%%%%%%%%%%%%%%%%%%%%%%%%%%%%%%
% 2. Necessary elements of XML metadata
%%%%%%%%%%%%%%%%%%%%%%%%%%%%%%%%%%%%%%%%%%%%%%%%%%%%%%%%%%%%%%%%%%%%%%%%%%%%%%%%%%%

\subsubsection*{2. Necessary elements of XML metadata with DTD}
\label{sec:mets}
{\bf Ruey-Shun}\cite{chen2003developing} suggest that an XML metadata discribed according the DTD include three necessary elements:
\begin{enumerate}
	\item Structure\\
	The major execution ability of structure includes parser for well-formed XML and
	valid DTD structure, authoring tool for editing.
	
	\item Depth\\
	Basically, there are two sorts of fields: Fixed-length fields and variable fields.
	Fixed-length fields are general types and character-indication types Sub-field, whether
	fixed-length fields or variable fields, might contain both fixed-length fields and
	variable fields. According to the reason above, the process ability of the system has to
	cover the situation
	
	\item Scope\\
	The connections must involve simple object, time, space, people, and event. 
\end{enumerate}


%%%%%%%%%%%%%%%%%%%%%%%%%%%%%%%%%%%%%%%%%%%%%%%%%%%%%%%%%%%%%%%%%%%%%%%%%%%%%%%%%
% The bibliography
%%%%%%%%%%%%%%%%%%%%%%%%%%%%%%%%%%%%%%%%%%%%%%%%%%%%%%%%%%%%%%%%%%%%%%%%%%%%%%%%%

	\bibliography{library}
	\clearpage 


%%%%%%%%%%%%%%%%%%%%%%%%%%%%%%%%%%%%%%%%%%%%%%%%%%%%%%%%%%%%%%%%%%%%%%%%%%%%%%%%%
% figures
%%%%%%%%%%%%%%%%%%%%%%%%%%%%%%%%%%%%%%%%%%%%%%%%%%%%%%%%%%%%%%%%%%%%%%%%%%%%%%%%%
\clearpage % Ends the current page and causes all figures and tables to be printed



\end{document}      % End of the document

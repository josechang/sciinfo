%%%%%%%%%%%%%%%%%%%%%%%%%%%%%%%%%%%%%%%%%%%%%%%%%%%%%%%%%%%%%%%%%%%%%%%%%%%%%%%%%%%
% Team:
% EagleUnit
% Members: 
% Chinweze Ubadigha, Feng-Chun Hsia, Henry Peng, I-Chieh Lin, Jones Hou, Piyarul Hoque, Ray Chang
% Relative files:
% Main.tex, Background_EagleUnit.tex, Library.bib, EagleUnit_Background_Chart_1.png
% Note:
% Do not compile this file compile Main.tex to get the pdf file instead.
%%%%%%%%%%%%%%%%%%%%%%%%%%%%%%%%%%%%%%%%%%%%%%%%%%%%%%%%%%%%%%%%%%%%%%%%%%%%%%%%%%%
\subsection{Web Server Structure}
By combining Django, PostgreSQL, NGINX, and Gunicorn, a robust web server was build.
The Django provide a very convenient web page framework to modify the outlooks and functions. 
For the database system, Postgres provide a much more powerful function and management then the default SQLite in Django.
By using Gunicor as an interface to the Django, NGINX makes the server more  fast and secured on communication with the internet requests.

\subsection{NGINX Server}
The NGINX server provide a particular internet domain to our web page,
which allowed us to connect the web page by using URL,
with out entering any port number. Additionally,
the NGINX server is not only playing the role of reverse proxy server of our Django server,
it can direct the user to any other domain in the main sever as well.
It means we can have other single HTML files not including in the Django server, or,
even an other Django project, and all linking to the same NGINX server.
NGINX can lead the user to projects according their own requests. 

\subsection{Web Page}
The final design of the search setting page has a search bar and several filters.
It allows users to enter the keywords, and decide which part of articles should the search system search in.
To display the articles found, we didn't shows them in the same page with search settings in the end. 
Instead, a link to a new page displaying all the metadata of search results was created.

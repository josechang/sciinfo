%%%%%%%%%%%%%%%%%%%%%%%%%%%%%%%%%%%%%%%%%%%%%%%%%%%%%%%%%%%%%%%%%%%%%%%%%%%%%%%%%%%
% Team:
% EagleUnit
% Members: 
% Chinweze Ubadigha, Feng-Chun Hsia, Henry Peng, I-Chieh Lin, Jones Hou, Piyarul Hoque, Ray Chang
% Relative files:
% Main.tex, Background_EagleUnit.tex, Library.bib, EagleUnit_Background_Chart_1.png
% Note:
% Do not compile this file compile Main.tex to get the pdf file instead.
%%%%%%%%%%%%%%%%%%%%%%%%%%%%%%%%%%%%%%%%%%%%%%%%%%%%%%%%%%%%%%%%%%%%%%%%%%%%%%%%%%%
\subsection{Webpage Construction}
\textit{\footnote size Author : Chinweze Ubadigha, Feng-Chun Hsia, Henry Peng, I-Chieh Lin, Jones Hou, Piyarul Hoque, Ray Chang.}\\

We are going to construct a public, user friendly web site, which provide service of scientific article searching.
To achieve this goal, we separate our responsibility into parts:\\
\begin{itemize}
	\item Build up the web server.
	The web server is basically built using Django web frame work on a CentOS 7 system. Django is a web framework which is constructed using Python code.
	For the purpose of this project the Django web server was built using PostgresSQL (database management application), 
	Nginx (traffic control and security), and 
	Gunicorn (interface to translate clients requests in HTTP to python calls that Django can understand)  
	and Python WSGI HTTP Server (used to create entry sock for Django)
	thus making a robust web server.
	\item Build a web page with a search bar.
	\item Place a list on the webpage which showing the search results.
	\item Build a metadata schema which contains all information in a PDF and then edit it in an XML structure. 
	\item Build up the APIs. The search bar should send the search string and the result list to tje search system 
	then the system receive the results of searching and display it.
\end{itemize}
This article will provide a review through the tools and methodologies we use to achieve each work.
\subsubsection{Web Server}
For any web page, a web server is necessary.
There are three web server system used in our project. 
They play the role as web page framework, reverse proxy server, 
and database management, respectively.
Consider the ease to use and learn, and the generality of Python language 
that we choose Django, a Python based web server system for our work.
\subsubsection{Ngnix}
Nginx is a high performance web server. It can act as an HTTP and reverse proxy server. When compared to Apache, Nginx is light-weight and has minimal hardware requirements to deal with web traffic. It has many features as follow:
\begin{itemize}
	\item Fastest and the best for serving static files
	\item Increased security of server 
	\item Handling many concurrent connections at the same time
	\item Load Balancing Support
\end{itemize}
\subsubsection{Django}
Django is one of the most well-known python web framework. There are many famous sites built with Django, such as Pinterest, Instagram, Disqus. It has many features as follow:
\begin{itemize}
	\item Free and open-source web framework
	\item Written in python
	\item MTV framework
	\item fast and secure
	\item Exceedingly scalable
	\item Fully loaded
	\item Incredibly versatile
\end{itemize}
Next, we are going to show how to install Django in Linux.
\begin{itemize}
	  
		\item Step 1\\
		Open the terminal and create folder named mydjango
		\begin{enumerate}
			\emph {\$ mkdir mydjango}\\
			\emph {\$ cd mydjango}
		\end{enumerate}	
		\item Step 2\\
		Create a vitual environment
		\begin{enumerate}
			\emph {/mydjango \$ virtualenv mydjangovenv}\\
			\emph {/mydjango \$ cd mydjangovenv}\\
			\emph {/mydjangovenv \$ source bin/activate}
		\end{enumerate}	
		\item Step 3\\
		Install django
		\begin{enumerate}
			\emph {(mydjangovenv) \$ python -m pip install django}\\
		\end{enumerate}	
\end{itemize}
\subsubsection{Django with Postgres, Nginx, and Gunicorn}

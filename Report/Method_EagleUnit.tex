%%%%%%%%%%%%%%%%%%%%%%%%%%%%%%%%%%%%%%%%%%%%%%%%%%%%%%%%%%%%%%%%%%%%%%%%%%%%%%%%%%%
% Team:
% EagleUnit
% Members: 
% Chinweze Ubadigha, Feng-Chun Hsia, Henry Peng, I-Chieh Lin, Jones Hou, Piyarul Hoque, Ray Chang
% Relative files:
% Main.tex, Background_EagleUnit.tex, Library.bib, EagleUnit_Background_Chart_1.png
% Note:
% Do not compile this file compile Main.tex to get the pdf file instead.
%%%%%%%%%%%%%%%%%%%%%%%%%%%%%%%%%%%%%%%%%%%%%%%%%%%%%%%%%%%%%%%%%%%%%%%%%%%%%%%%%%%
\subsection{Webpage Construction}
\textit{\footnotesize Author : Chinweze Ubadigha, Feng-Chun Hsia, Henry Peng, I-Chieh Lin, Jones Hou, Piyarul Hoque, Ray Chang.}\\

We are going to construct a public, user friendly web site, which provide service of scientific article searching.
To achieve this goal, we seperate our responsibility into parts:\\
\begin{itemize}
	\item Build a webpage with a search bar.
	\item Place a list on the webpage which showing the search results.
	\item Build a metadata schema which contains all information in a PDF and then edit it in an XML structure. 
	\item Build up the APIs. The search bar should send the search string to tje search system, and the result list then recieve the results of searching and display it.
\end{itemize}
\subsubsection{Web Server}
To construct a public web page, a web server will be necessary. Consider the ease to use and learn, and the genarality of Python language, we choose Django, a Python based web server system for our work.


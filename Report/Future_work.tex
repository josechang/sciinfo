%%%%%%%%%%%%%%%%%%%%%%%%%%%%%%%%%%%%%%%%%%%%%%%%%%%%%%%%%%%%%%%%%%%
% Future Work
% Year:
% 2017
% Team:
% RCPL
% Members: 
% Lewis Hsu, Paul Lin, Tam-Van Ngo
% Relative files:
% Future_work.tex
% Note:    
% Do not compile this file compile Main.tex to get the pdf file instead.
%%%%%%%%%%%%%%%%%%%%%%%%%%%%%%%%%%%%%%%%%%%%%%%%%%%%%%%%%%%%%%%%%%%

\subsection{Future Work}
So far, our web page have implemented full text search function. After clicking the search icon, it can perform full text search through the PDF files in the database and extracts of the most similar parts in decreasing order. Besides, it can also diplay the title and its relevance indescending order  on the result web page.

However, we did not perform the search function through a PDF reader. Instead, we provide a search box on a web page which the user cannot click on a highlighted text and thereby perform a full text search through the PDF reader. Therefore, for next step, it will be great to build or find a PDF reader which the strings inside the text can be clicked. Then, it will perform the full text search through the PDF files in the database. The current search process took quite a bit of time under a very small size of testing samples(89 articles), the response time of the system would be much more slower when the size of the database increase to the 10,000 articles. Therefore, optimization of database, search function and web server is needed. This will enhance the user experience while searching. We also hope that more search methods can be considered and implemented to the system. Last but not least, visualization of result is needed. The website should provide charts about the distribution of similarity score of different parts of article to the search query, also provide service to show the wanted result by selecting the relevent boxes(ex. rank by publishing year, show result based on different types of articles).
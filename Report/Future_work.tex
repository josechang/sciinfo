%%%%%%%%%%%%%%%%%%%%%%%%%%%%%%%%%%%%%%%%%%%%%%%%%%%%%%%%%%%%%%%%%%%
% Future Work
% Year:
% 2017
% Team:
% RCPL
% Members: 
% Lewis Hsu, Paul Lin, Tam-Van Ngo
%Team:
%Rainy
%Dickson, Rain, Sareddy
% Relative files:
% Future_work.tex
% Note:    
% Do not compile this file compile Main.tex to get the pdf file instead.
%%%%%%%%%%%%%%%%%%%%%%%%%%%%%%%%%%%%%%%%%%%%%%%%%%%%%%%%%%%%%%%%%%%

\subsection{Future Work}
So far, our web page has  implemented the full-text search function. After clicking the search icon, it can perform the full-text search through the PDF files in the database and extracts of the most similar parts in decreasing order. Besides, it can also display the title and its relevance in descending order  on the result web page.
However, we did not perform the search function through a PDF reader. Instead, we provide a search box on a web page which the user cannot click on a highlighted text and thereby perform a full-text search through the PDF reader. Therefore, it would be great to build or find a PDF reader which the strings inside the text can be clicked and perform full-text search through the PDF files in the database as the next step of the future work. The current search process took quite a bit of time  to  a very small size of testing samples(89 articles), the response time of the system would be much more slower when the size of database increases. Therefore, optimization of  a database, search function and the web server is needed. This will enhance the user experience while searching. We also hope that more searching methods can be considered and implemented to the system. Apart from the performance of the web service, it is also worth providing a  hyper link to the original source of PDF file and using Harvard citation format.
Last but not least, visualization of the result is needed. The website should provide charts about the distribution of similarity score of different parts of  the article to the search query, also provide service to show the wanted result by selecting the relevant boxes(ex. rank by publishing year, show result based on different types of articles etc).
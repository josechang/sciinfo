\documentclass{article}
\usepackage[utf8]{inputenc}

\title{Summary of Sanderson M and Croft WB 2012}
\author{Michael Huang \\N16064674 }
\date{April 1 2018}
\addtolength{\topmargin}{-30mm}% Removes 30mm from the top margin
\addtolength{\textheight}{30mm}
\begin{document}

\maketitle

\section{Problem}
The article focus on the history of the information retrieval (IR). Starting from the electromechanical devices to computer-based search engine. The review paper finished with a perspective on the future challenge for information retrieval.
\section{Methods}
Early use of computers on information retrieval is indexing and ranking system which provide a brief introduction to certain article. Then it led to the development of term frequency which shows how often a specific phrase appears in the paper. In the late 20th century, the help of semantic and synonyms gave search engine a more flexible approach. Recently, query log combined with Boolean search was widely used.
\section{Conclusion}
The accessibility of information has improved during 20th and 21st century. People who needed information could only acquire a limited number of the relevant articles in early 20th century. In the 21st century, with the help of internet, people can access countless web pages with articles, video clips or photos within a minute searching time. The author said that the completeness of IR system requires much innovation and thought over a long period of time and it still have a long route to accomplish.



%%%%%%%%%%%%%%%%%%%%%%%%%%%%%%%%%%%%%%%%%%%%%%%%%%%%%%%%%%%%%%%%%%%%%%%%%%%%%%%%%
%                               THE BIBLIOGRAPHY                                %
%%%%%%%%%%%%%%%%%%%%%%%%%%%%%%%%%%%%%%%%%%%%%%%%%%%%%%%%%%%%%%%%%%%%%%%%%%%%%%%%%
% Add bibliography manually

\begin{thebibliography}{99}

\bibitem{Sanderson2012}
SANDERSON, Mark; CROFT, W. Bruce. "The history of information retrieval research." 
Proceedings of the IEEE, 100.Special Centennial Issue: 1444-1451 (2012).



\end{thebibliography}


\end{document}
